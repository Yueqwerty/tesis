\begin{figure}[htbp]
    \centering
    \begin{tikzpicture}[
        font=\small,
        node distance=1.2cm,
        process/.style={
            rectangle, draw=black, line width=1pt,
            minimum height=1cm, minimum width=3.5cm,
            text centered, font=\small, align=center
        },
        stochastic/.style={
            rectangle, draw=black, line width=1pt,
            minimum height=1.2cm, minimum width=3.5cm,
            fill=gray!15, text centered, font=\small, align=center
        },
        arrow/.style={->, >=stealth, line width=1pt},
        label/.style={font=\footnotesize, fill=white, inner sep=1pt}
    ]

        % Nodo inicial
        \node[process] (inicio) at (0,0) {Día 0};

        % Generar tiempo entre eventos
        \node[stochastic, below=of inicio] (tiempo) {
            Generar $T$\\
            $T \sim \text{Exp}(\lambda = 4/\text{año})$
        };

        % Espera
        \node[process, below=of tiempo] (espera) {Esperar $T$ días};

        % Generar duración
        \node[stochastic, below=of espera] (duracion) {
            Generar $D$\\
            $D \sim \text{Tri}(a, b, c)$
        };

        % Bloquear ruta
        \node[process, below=of duracion] (bloquear) {
            Bloquear Ruta\\
            $D$ días
        };

        % Registro
        \node[process, below=of bloquear] (registro) {Registrar disrupción};

        % Flechas
        \draw[arrow] (inicio) -- (tiempo);
        \draw[arrow] (tiempo) -- (espera);
        \draw[arrow] (espera) -- (duracion);
        \draw[arrow] (duracion) -- (bloquear);
        \draw[arrow] (bloquear) -- (registro);

        % Loop de regreso
        \draw[arrow] (registro.east) -- ++(1.2,0) node[label, right] {Repetir} |- (tiempo.east);

    \end{tikzpicture}
    \caption{Diagrama de flujo del proceso de generación de disrupciones.}
    \label{fig:generacion-disrupciones}
\end{figure}
