\chapter{Hipótesis y Pregunta de Investigación}
\label{chap:hipotesis}

El análisis del problema, expuesto en el \cref{chap:planteamiento-problema}, 
revela un sistema sociotécnico complejo, dominado por la interacción de 
variables estocásticas y decisiones estratégicas. La disonancia entre la 
magnitud de las amenazas exógenas y la limitada capacidad de respuesta 
endógena sugiere que no todos los parámetros del sistema contribuyen de 
igual manera a su vulnerabilidad global. Para abordar esta complejidad de 
manera científica, es imperativo formular una pregunta de investigación que 
guíe el diseño de una herramienta de análisis cuantitativo, y una hipótesis 
falsificable que postule una relación de sensibilidad entre los factores 
críticos del sistema.

\section{Pregunta de Investigación}
\label{sec:pregunta-investigacion}

Considerando la caracterización del sistema, sus vulnerabilidades y la
brecha metodológica identificada en los marcos de análisis actuales, la
pregunta central que esta investigación busca responder es:
\begin{quote}
    \textit{¿Cómo diseñar, implementar y validar un modelo de simulación de
    eventos discretos para cuantificar el impacto de parámetros logísticos
    críticos ---tales como la duración de las disrupciones de ruta y las
    políticas de inventario--- sobre la resiliencia del sistema de suministro
    de GLP en Aysén?}
\end{quote}
Esta pregunta aborda tanto la construcción del modelo como la validación de
su utilidad para analizar este sistema sociotécnico.

\section{Hipótesis Central}
\label{sec:hipotesis}

Como una respuesta conjetural y testable a la pregunta de investigación, y 
derivada directamente de la disonancia identificada en el planteamiento del 
problema, se postula la siguiente hipótesis central:
\begin{quote}
    \textbf{Hipótesis:} La resiliencia del sistema de suministro de GLP de 
    Aysén, cuantificada a través de métricas de rendimiento como el Nivel de 
    Servicio, exhibe una sensibilidad significativamente mayor a la 
    variabilidad de los parámetros exógenos que a la de los parámetros 
    endógenos. Específicamente, se postula que la elasticidad de la 
    resiliencia con respecto a la duración de las disrupciones de ruta 
    es superior a su elasticidad con respecto a las variaciones en la 
    capacidad de almacenamiento primario.
\end{quote}
Esta hipótesis formaliza la intuición de que la magnitud de las amenazas 
externas (ej. un corte de 21 días) es el factor dominante que gobierna el 
comportamiento del sistema, por sobre las mejoras incrementales en la 
capacidad de respuesta interna (ej. aumentar el inventario de 8 a 12 días 
de autonomía). El diseño experimental, detallado en el 
\cref{chap:metodologia}, está específicamente concebido para proveer la 
evidencia estadística necesaria para confirmar o refutar esta afirmación, 
jerarquizando así las palancas de acción más efectivas para fortalecer la 
seguridad energética regional.