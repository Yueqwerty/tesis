\chapter{Planteamiento del Problema}
\label{chap:planteamiento-problema}

La Región de Aysén enfrenta un riesgo sistémico en su suministro de GLP. Este capítulo caracteriza el sistema (demanda, infraestructura, actores), identifica sus vulnerabilidades (exógenas y endógenas), y argumenta por qué los métodos de análisis actuales (matrices de riesgo estáticas) son insuficientes para evaluar la resiliencia dinámica del sistema.

\section{Caracterización del Sistema}
\label{sec:caracterizacion-sistema}

El sistema de suministro de GLP en Aysén opera bajo condiciones que lo
definen como un sistema complejo y crítico. Su criticidad se deriva de la
confluencia de una demanda energética elevada, una estructura de mercado
oligopólica y una infraestructura precaria.

\subsection{Demanda energética y crecimiento proyectado}

La demanda energética de la región es un factor distintivo a nivel nacional. 
Con un consumo per cápita de \SI{27.25}{Gcal}, que excede en un 65\% la 
media de Chile, la población depende intensivamente de un flujo energético 
constante para sostener funciones básicas. El GLP satisface nichos de 
consumo ---cocción de alimentos y agua caliente sanitaria--- para los cuales 
no existen sustitutos inmediatos a gran escala. Esta inelasticidad 
fundamental de la demanda convierte al GLP en un recurso de alta prioridad 
social, cuya interrupción tiene consecuencias directas e inmediatas sobre 
el bienestar de la población.

Adicionalmente, la demanda no es estática. El sistema enfrenta un 
crecimiento sostenido del \SI{3.8}{\percent} anual, impulsado por el 
desarrollo demográfico y económico de la región. A esto se suma un factor 
disruptivo futuro: la proyectada incorporación de una central térmica que 
consumirá \SI{14.4}{ton/día} de GLP. Este crecimiento y cambio estructural 
implican que la presión sobre la ya frágil cadena de suministro se 
intensificará de manera no lineal en los próximos años.

\subsection{Estructura de la cadena de suministro}

% --- SECCIÓN PROFUNDIZADA CON ANÁLISIS DE LA RUTA ---
La estructura de la cadena de suministro, detallada en el informe de 
referencia~\cite{CIEP2025}\footnote{El informe, encargado por la Seremía de 
Energía de Aysén, fue finalizado y presentado en 2025. A la fecha de esta 
publicación, se encuentra en proceso de difusión pública.}, se caracteriza
por una topología logística 
lineal con una dependencia absoluta de fuentes exógenas. El informe es 
categórico al afirmar que ``La totalidad del gas licuado que llega a la 
región, lo hace vía camiones que transitan por el paso Huemules desde 
Argentina''. Este diseño, carente de redundancia, convierte a la ruta 
terrestre en un punto único de falla (\textit{Single Point of Failure}) de 
manual. La magnitud de esta dependencia se cuantifica en el hecho de que el 
86\% del recorrido terrestre desde la planta de Cabo Negro hasta Coyhaique 
transcurre por territorio argentino, lo que implica que la soberanía 
logística y la seguridad del suministro regional están estructuralmente 
comprometidas por la dinámica de un país vecino.

\section{Análisis de vulnerabilidades}
\label{sec:analisis-vulnerabilidades}

El sistema presenta vulnerabilidades tanto exógenas (amenazas externas) como endógenas (limitaciones de capacidad interna).

\subsection{Vulnerabilidades exógenas}

% --- SECCIÓN PROFUNDIZADA CON CUANTIFICACIÓN DEL RIESGO ---
El principal vector de riesgo es la incertidumbre estocástica en el 
aprovisionamiento. El informe de riesgos~\cite{CIEP2025} identifica dos 
amenazas dominantes. Primero, el evento de ``Nevadas / Cierre cruce 
fronterizo'' (Tag \#57), clasificado con una probabilidad de Nivel 4 (``Casi 
Seguro''), lo que implica una frecuencia de ocurrencia de ``al menos una vez 
cada tres meses''. Segundo, el riesgo de ``Conflicto social en Argentina'' 
(Tag \#45), que, aunque menos frecuente, ha demostrado la capacidad de 
paralizar el suministro por períodos de hasta tres semanas. Esta 
realidad es corroborada por la autoridad energética regional, quien 
identifica la ``conflictividad social'' y la ``mantención de los caminos'' 
en Argentina como la principal amenaza para la continuidad del suministro 
(Laibe, T., comunicación personal, 11 de junio, 2025). El sistema, por tanto, 
enfrenta amenazas recurrentes y de duración prolongada que introducen una 
alta variabilidad en el tiempo de entrega (\textit{lead time}).

\subsection{Vulnerabilidades endógenas}

La capacidad de almacenamiento en Coyhaique es de 431 TM (Abastible 150, Lipigas 240, Gasco 41)~\cite{CIEP2025}, proporcionando 8,2 días de autonomía vs. disrupciones documentadas de hasta 21 días.

El informe CIEP 2025 identifica que "Gasco tiene una capacidad demasiado reducida para su nivel de ventas" (41 TM = 0,78 días de autonomía). Esta situación refleja un Dilema del Prisionero en el mercado oligopólico: minimizar costos de inventario individualmente (estrategia racional) degrada la resiliencia colectiva.

\textbf{Propuesta 10.4 (Gasco):} El informe documenta que Gasco está desarrollando un proyecto de expansión para incrementar su capacidad en 250 TM (de 41 a 291 TM), elevando la capacidad total del sistema de 431 TM a 681 TM. La inversión estimada es \$1,5 millones USD. Esta expansión aumentaría la autonomía teórica del sistema de 8,2 a 13 días.

El presente trabajo evalúa cuantitativamente el retorno en resiliencia de esta inversión bajo diferentes escenarios de disrupciones.

\section{Limitaciones del diagnóstico actual}
\label{sec:insuficiencia-marcos}

\subsection{Vacío en la planificación energética regional}

La 'Política Energética 2050 Región de Aysén' se enfoca exclusivamente en el sector eléctrico, omitiendo la cadena de suministro de combustibles.

\subsection{Análisis estático vs. dinámico}

El informe CIEP 2025 es exhaustivo pero estático: identifica riesgos mediante una matriz de probabilidad × impacto, pero no puede cuantificar:

\begin{itemize}
    \item ¿Cuánto tiempo puede operar el sistema bajo disrupciones largas?
    \item ¿Qué tan frecuentemente fallará el sistema (quiebres de stock)?
    \item ¿Cuál es el retorno en resiliencia de la inversión de \$1,5M USD (Propuesta 10.4)?
\end{itemize}

\subsection{Gestión reactiva de crisis}

La gestión operativa de crisis se basa en protocolos manuales: "llamar a todas las empresas, construir un Excel y sacar una foto de cuánta disponibilidad de combustible hay" (Laibe, T., comunicación personal, 11 de junio, 2025). No existe una herramienta de análisis dinámico.

\subsection{Pregunta que el diagnóstico actual no puede responder}

\textit{¿Cuál es la reducción en probabilidad de quiebre de stock que genera la inversión de \$1,5M USD (Propuesta 10.4 de Gasco), bajo un escenario de disrupciones con frecuencia 4 eventos/año y duración estocástica de hasta 21 días?}

Esta tesis desarrolla el modelo para responder esta pregunta.