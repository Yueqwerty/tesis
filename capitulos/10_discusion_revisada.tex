\chapter{Discusión}
\label{chap:discusion}

Este capítulo interpreta los resultados del experimento Monte Carlo. La sección \ref{sec:interpretacion-teorica} analiza la sensibilidad relativa de los factores endógenos y exógenos. La sección \ref{sec:limitaciones} identifica las simplificaciones del modelo y su impacto en las conclusiones. La sección \ref{sec:trabajo-futuro} propone extensiones para investigación futura.

\section{Interpretación de los Hallazgos}
\label{sec:interpretacion-teorica}

\subsection{Dominancia del factor endógeno: resultado contraintuitivo}

El experimento factorial REFUTÓ la hipótesis inicial. Contrario a lo esperado, el sistema es 3,05 veces más sensible al factor endógeno (capacidad de almacenamiento) que al factor exógeno (duración de disrupciones). Específicamente:

\begin{itemize}
    \item \textbf{Efecto endógeno:} Incrementar capacidad de 431 TM a 681 TM (+58\%) mejora el nivel de servicio en 15,72 puntos porcentuales (de 81,20\% a 96,91\%).
    \item \textbf{Efecto exógeno:} Incrementar duración máxima de disrupciones de 7 a 21 días degrada el nivel de servicio en 5,15 puntos porcentuales (de 91,57\% a 86,42\%).
\end{itemize}

\textbf{Ratio de sensibilidad:} $15,72 / 5,15 = 3,05$

Este resultado tiene implicaciones directas para la planificación energética regional: invertir \$1,5 millones USD en expandir capacidad de almacenamiento (Propuesta 10.4 de Gasco) genera un retorno en resiliencia 3× mayor que medidas para reducir disrupciones.

\subsection{Nivel de servicio del sistema actual: subcapacidad crónica}

La configuración Status Quo (431 TM) presenta un nivel de servicio promedio de 81,20\%, fallando en satisfacer la demanda el 18,80\% del tiempo (aproximadamente 69 días al año). Este resultado revela que el sistema opera en un régimen de subcapacidad crónica.

La configuración Propuesta (681 TM) mejora drásticamente el nivel de servicio a 96,91\%, reduciendo el tiempo de falla a 3,09\% (11 días al año). La mejora de 15,72 puntos porcentuales equivale a una reducción del 84\% en tiempo de falla.

\subsection{Explicación del comportamiento: umbral crítico de capacidad}

Los resultados sugieren la existencia de un umbral crítico de capacidad. El Status Quo (431 TM) opera por debajo de este umbral: la capacidad es insuficiente para absorber la variabilidad estocástica de la demanda (modelada con $\pm$15\% de ruido), generando quiebres de stock frecuentes incluso sin disrupciones prolongadas.

Con demanda base de 52,5 TM/día (mes de mayor consumo):
\begin{itemize}
    \item \textbf{Autonomía Status Quo:} $431 / 52.5 = 8,2$ días
    \item \textbf{Autonomía Propuesta:} $681 / 52.5 = 13,0$ días
\end{itemize}

La Propuesta cruza el umbral operativo, permitiendo absorber fluctuaciones de demanda y disrupciones moderadas (7-14 días). Sin embargo, disrupciones de 21 días aún generan quiebres (nivel de servicio 94,70\% en escenario Propuesta-Larga).

\section{Alcance y rango de validez del modelo}
\label{sec:interpretacion-modelo}

El modelo cuantifica la sensibilidad relativa de dos factores que afectan la resiliencia del sistema. Los resultados demuestran que bajo las condiciones operacionales actuales, el factor endógeno (capacidad de almacenamiento) tiene mayor impacto que el exógeno (duración de disrupciones) en una proporción de 3,05:1.

\subsection{Parámetros del modelo}

Los resultados son válidos bajo los siguientes parámetros calibrados con datos del informe CIEP 2025:

\begin{itemize}
    \item \textbf{Capacidad:} Status Quo 431 TM (Abastible 150, Lipigas 240, Gasco 41), Propuesta 681 TM.
    \item \textbf{Política de inventario:} $(Q,R)$ con $R=50\%$ capacidad, $Q=50\%$ capacidad.
    \item \textbf{Demanda:} Base 52,5 TM/día (mes de mayor consumo) + variabilidad estocástica $\pm$15\% + estacionalidad $\pm$25\%.
    \item \textbf{Disrupciones:} Frecuencia Poisson $\lambda=4$ eventos/año, duración Triangular(3, modo, 7-21) días.
    \item \textbf{Lead time nominal:} 6 días (1.400 km desde Cabo Negro/Neuquén).
    \item \textbf{Experimento:} 10.000 réplicas por configuración, 60.000 simulaciones totales.
\end{itemize}

\subsection{Limitaciones de la demanda base}

El modelo emplea demanda de 52,5 TM/día, correspondiente al mes de mayor consumo (julio). Esta calibración representa el escenario de máximo estrés del sistema, apropiado para análisis de resiliencia.

Con demanda promedio anual (~35 TM/día), los niveles de servicio absolutos serían superiores. Sin embargo, el ratio de sensibilidad 3,05× se mantendría aproximadamente constante al ser una medida relativa. Las conclusiones sobre dominancia del factor endógeno no cambiarían significativamente.

\section{Limitaciones del Estudio}
\label{sec:limitaciones}

\subsection{Simplificaciones del modelo}

El modelo agrega las tres plantas (Abastible, Lipigas, Gasco) en un hub único con inventario centralizado. No diferencia entre rutas de abastecimiento (Cabo Negro vs. Neuquén) ni modela dinámicas competitivas entre distribuidores.

Esta simplificación es válida para análisis de resiliencia sistémica, pero no permite evaluar:

\begin{itemize}
    \item Quiebres de stock diferenciados por distribuidor (ej. Gasco con solo 41 TM de capacidad).
    \item Vulnerabilidades de localidades remotas fuera de Coyhaique (Chile Chico, Cochrane).
    \item Beneficios de diversificar fuentes de aprovisionamiento o rutas alternativas.
\end{itemize}

\subsection{Horizonte temporal de 1 año}

Cada simulación cubre 365 días. Un horizonte de 5-10 años permitiría evaluar:

\begin{itemize}
    \item Crecimiento de demanda (3,8\% anual proyectado) + nueva central térmica (14,4 TM/día).
    \item Cambios en frecuencia de disrupciones por variabilidad climática.
    \item Degradación de infraestructura vial (Ruta 7) y costos de mantenimiento.
\end{itemize}

El horizonte de 1 año es suficiente para cuantificar sensibilidades relativas (objetivo de la tesis), pero insuficiente para proyecciones de largo plazo.

\subsection{Datos de entrada}

Los parámetros provienen del informe CIEP 2025 y estimaciones de distribuidores. No se dispone de:

\begin{itemize}
    \item Series temporales de inventario real por distribuidor.
    \item Registro histórico completo de disrupciones (fechas, duraciones, causas).
    \item Datos de demanda horaria o diaria (solo promedios mensuales).
\end{itemize}

La frecuencia de disrupciones (4 eventos/año) se basa en la matriz de riesgos del informe, no en datos empíricos de años anteriores. Validación con datos históricos mejoraría la precisión del modelo.

\section{Trabajo futuro}
\label{sec:trabajo-futuro}

\subsection{Modelo multi-agente por distribuidor}

Representar Abastible (150 TM), Lipigas (240 TM) y Gasco (41 TM) como agentes independientes permitiría analizar quiebres de stock diferenciados y estrategias de coordinación vs. competencia.

\subsection{Optimización de política $(Q,R)$}

Determinar parámetros óptimos de $(Q,R)$ que minimicen costo de inventario + costo de quiebres, o evaluar políticas adaptativas que ajusten $R$ según pronóstico de disrupciones.

\subsection{Rutas alternativas y mitigación de disrupciones}

Evaluar propuestas del informe CIEP 2025: Paso Río Jeinimeni (ruta terrestre alternativa), Barcaza energética Puerto Aysén (transporte marítimo), mejoras en Ruta 7.

\subsection{Validación con datos históricos}

Acceso a series temporales de inventario diario (2019-2024), registro de disrupciones (fechas, duraciones, causas), y datos de demanda horaria permitiría calibración empírica y validación predictiva del modelo.

\subsection{Proyecciones de largo plazo (5-10 años)}

Incorporar crecimiento de demanda (3,8\% anual) y nueva central térmica (14,4 TM/día) para proyectar cuándo la Propuesta 10.4 de Gasco (681 TM) será insuficiente.
