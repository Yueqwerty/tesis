\chapter{Introducción}
\label{chap:introduccion}

El suministro de recursos energéticos en regiones aisladas enfrenta desafíos
particulares debido a la variabilidad de las condiciones operacionales y la
necesidad de asegurar la continuidad del servicio. Estos desafíos requieren
herramientas analíticas que consideren la incertidumbre inherente del sistema.
La Región de Aysén, en la Patagonia chilena, ilustra esta problemática en su
sistema de suministro de Gas Licuado de Petróleo (GLP).

Este trabajo aborda la vulnerabilidad de dicha cadena de suministro. El
objetivo es diseñar y validar un modelo de simulación de eventos discretos
para analizar cuantitativamente la resiliencia del sistema. El modelo se
enfoca en el nodo logístico de Coyhaique, punto central del sistema de
distribución regional, y permite evaluar la interacción de parámetros
logísticos, cuantificar el impacto de disrupciones y evaluar estrategias de
mitigación.

La investigación se estructura de manera deductiva para guiar al lector desde 
el contexto general hasta los detalles técnicos y los resultados. El 
\cref{chap:planteamiento-problema} contextualiza y justifica la relevancia 
del problema. El \cref{chap:hipotesis} formaliza la pregunta de 
investigación y la hipótesis central que guían el estudio. El 
\cref{chap:objetivos} delimita las metas del proyecto de manera precisa. Los 
capítulos \ref{chap:marco-teorico} y \ref{chap:estado-del-arte} establecen 
los fundamentos teóricos y el estado del arte en la materia. Finalmente, el 
\cref{chap:metodologia} detalla el diseño metodológico para el desarrollo y 
la validación del prototipo.