\chapter{Resultados del Experimento Computacional}
\label{chap:resultados}

Este capítulo presenta los datos obtenidos tras la ejecución masiva del simulador. Se analiza el comportamiento del sistema a partir de los datos sintéticos generados en un experimento de Monte Carlo compuesto por 100.000 iteraciones independientes.

\section{Análisis del Escenario Base (Status Quo)}

La primera fase del experimento evaluó la configuración actual de infraestructura ($K=431$ toneladas). El análisis de las series de tiempo generadas revela un comportamiento bimodal en el sistema.

\subsection{Modo de Operación Nominal}
En las iteraciones donde la variable aleatoria de corte de ruta tomó valores bajos ($T_{corte} \leq 7$ días), el sistema mantuvo la estabilidad. El algoritmo de reabastecimiento logró compensar el consumo diario, manteniendo el nivel del tanque oscilando dentro de la banda de seguridad. El Nivel de Servicio promedio registrado fue de 99,2\%, con una varianza mínima.

\subsection{Modo de Falla Catastrófica}
Cuando el generador estocástico introdujo eventos de larga duración ($T_{corte} \approx 21$ días), se observó un desacople del sistema. La tasa de consumo ($\approx 52$ ton/día) agotó el inventario integral ($I(t)=0$) antes de que la ruta fuera liberada. Matemáticamente, el tiempo de recuperación del inventario se volvió asintótico, resultando en un periodo medio de 8 días de demanda insatisfecha no recuperable.

\section{Análisis de Sensibilidad Paramétrica}

Para cuantificar la elasticidad del sistema ante cambios en sus variables de diseño, se ejecutó un análisis de sensibilidad comparando dos factores: la capacidad de almacenamiento $K$ y la tasa de reparación de ruta $\mu$.

Los resultados numéricos indican que la elasticidad del Nivel de Servicio respecto a la Capacidad ($K$) es significativamente mayor que respecto a la gestión de la ruta. Al incrementar el parámetro $K$ en un 50\% (simulando la inversión en infraestructura), la probabilidad de quiebre de stock bajo el escenario de estrés máximo disminuyó en un orden de magnitud.

\section{Conclusión del Análisis}

Desde una perspectiva de ingeniería de sistemas, los resultados confirman que el sistema actual carece de redundancia para tolerar fallas de modo común en el canal de transporte. La solución óptima, basada en los datos de simulación, es desacoplar las constantes de tiempo del sistema: aumentar la capacidad de almacenamiento local incrementa la constante de tiempo de descarga del inventario, permitiendo que el sistema ``sobreviva'' a periodos de desconexión más largos sin degradar el servicio al cliente final.