\chapter{Marco Teórico}
\label{chap:marco-teorico}

El diseño de un modelo de simulación para analizar un sistema sociotécnico
complejo requiere un fundamento dual. Primero, establecer el marco conceptual
que gobierna la dinámica del sistema real. Segundo, formalizar la metodología
de análisis para estudiar dicho sistema. Este capítulo establece ambos
fundamentos, proporcionando las bases teóricas y técnicas para el diseño del
modelo de simulación y la interpretación de sus resultados.

\section{Teorías del Dominio del Problema: La Cadena de Suministro}
\label{sec:scm-theory}

\subsection{Dinámica y Control de Inventarios en Entornos Estocásticos}
El núcleo de cualquier cadena de suministro es la gestión de su inventario. 
La dinámica del nivel de inventario \(I\) en una planta a lo largo del 
tiempo \(t\) puede ser descrita por la ecuación fundamental de balance:
$$ I(t) = I(t-1) - D(t) + A(t) $$
donde \(D(t)\) es la demanda agregada durante el período \(t\), y \(A(t)\) 
es la cantidad de producto que llega (arribos) en el mismo período. En un 
entorno estocástico como el de Aysén, tanto la demanda \(D(t)\) como, y de 
forma más crítica, el tiempo de entrega (\textit{lead time}, \(LT\)) que 
determina los arribos \(A(t)\), son variables aleatorias.

Para gestionar esta incertidumbre, la teoría de control de inventarios 
establece políticas de reabastecimiento. Para este estudio, se adopta un 
modelo de revisión continua \((Q, R)\), donde se ordena una cantidad fija 
\(Q\) cada vez que el inventario alcanza un Punto de Reorden \(R\).

\begin{figure}[htbp]
    \centering
    \begin{tikzpicture}[
        font=\footnotesize,
        axis/.style={->, >=Stealth, thick, black},
        inventory/.style={thick, blue!80, line width=2pt},
        reference/.style={dashed, gray!60, line width=1pt},
        leadtime/.style={<->, red!70, thick, line width=1.5pt},
        annotation/.style={font=\tiny, align=center}
    ]
        % Definir parámetros
        \def\ss{1.2}
        \def\rop{2.4}
        \def\maxinv{4.8}
        \def\minlevel{0.8}
        
        % Ejes principales
        \draw[axis] (0,0) -- (12,0) node[below right] {\textbf{Tiempo}};
        \draw[axis] (0,0) -- (0,5.5) node[above left, rotate=90, anchor=south] {\textbf{Nivel de Inventario}};
        
        % Líneas de referencia con mejor estilo
        \draw[reference] (0,\ss) -- (11.5,\ss);
        \node[right, black, font=\small] at (11.5,\ss) {Stock de Seguridad (SS)};
        
        \draw[reference] (0,\rop) -- (11.5,\rop);
        \node[right, black, font=\small] at (11.5,\rop) {Punto de Reorden (R)};
        
        % Curva de inventario mejorada
        % Primer ciclo
        \draw[inventory] (0.5,\maxinv) -- (4,\rop);
        \draw[inventory, dashed] (4,\rop) -- (6,\minlevel);
        
        % Llegada de orden 1
        \draw[inventory] (6,\minlevel) -- (6,\maxinv);
        \draw[red, thick, ->] (6,0.3) -- (6,\minlevel) node[midway, right, red, annotation] {Llegada};
        
        % Segundo ciclo
        \draw[inventory] (6,\maxinv) -- (9,\rop);
        \draw[inventory, dashed] (9,\rop) -- (11,\minlevel);
        
        % Llegada de orden 2
        \draw[red, thick, ->] (11,0.3) -- (11,\minlevel) node[midway, right, red, annotation] {Llegada};
        
        % Indicadores de emisión de órdenes
        \fill[red] (4,\rop) circle (2.5pt);
        \node[below, red, annotation] at (4,\rop-0.3) {Se emite orden 1};
        
        \fill[red] (9,\rop) circle (2.5pt);
        \node[below, red, annotation] at (9,\rop-0.3) {Se emite orden 2};
        
        % Lead times con mejor visualización
        \draw[leadtime] (4,0.4) -- (6,0.4) node[midway, below, red!70] {\(LT_1\)};
        \draw[leadtime] (9,0.4) -- (11,0.4) node[midway, below, red!70] {\(LT_2\)};
        
        % Líneas verticales de referencia
        \draw[gray, thin, dotted] (4,0) -- (4,\rop);
        \draw[gray, thin, dotted] (6,0) -- (6,\maxinv);
        \draw[gray, thin, dotted] (9,0) -- (9,\rop);
        \draw[gray, thin, dotted] (11,0) -- (11,\maxinv);
        
        % Etiquetas de cantidades en el eje Y
        \node[left, font=\small] at (0,\maxinv) {\(Q\)};
        \node[left, font=\small] at (0,\rop) {\(R\)};
        \node[left, font=\small] at (0,\ss) {\(SS\)};
        
        % Anotación del consumo
        \draw[blue!60, <->] (2,\maxinv-0.2) -- (2,\rop+0.2) 
            node[midway, right, blue!80, annotation, text width=1.8cm] {Consumo normal \(D(t)\)};
            
        % Anotación del consumo durante lead time
        \draw[red!60, <->] (4.8,\rop) -- (4.8,\minlevel) 
            node[midway, right, red!70, annotation, text width=2cm] {Consumo durante \textit{lead time}};
            
        % Área sombreada para stock de seguridad
        \fill[blue!5] (0,0) rectangle (11.5,\ss);
        \node[blue!60, annotation] at (5.75,0.6) {Zona de Stock de Seguridad};
        
    \end{tikzpicture}
    \caption{Modelo de inventario $(Q,R)$ con Punto de Reorden (ROP) y Stock de Seguridad (SS) bajo incertidumbre en el \textit{lead time}.}
    \label{fig:inventory-model-detailed}
\end{figure}

El cálculo del Punto de Reorden se formaliza como:
$$ R = (\bar{D} \times \bar{LT}) + SS $$
donde \(\bar{D}\) es la demanda promedio y \(\bar{LT}\) es el \textit{lead 
time} promedio. El componente crucial es el Stock de Seguridad (\(SS\)), el 
buffer que protege contra la variabilidad. Se calcula como:
$$ SS = Z_{\alpha} \times \sqrt{\bar{LT}\sigma_D^2 + \bar{D}^2\sigma_{LT}^2} $$
donde \(Z_{\alpha}\) es el factor de servicio para un nivel de servicio 
deseado \(\alpha\), \(\sigma_D\) es la desviación estándar de la demanda, y 
\(\sigma_{LT}\) es la desviación estándar del \textit{lead time}. Dado que 
en Aysén la demanda es relativamente estable pero el \textit{lead time} es 
altamente volátil, la ecuación se simplifica y evidencia que 
\(SS \propto \sigma_{LT}\). Esto formaliza matemáticamente la vulnerabilidad 
central del sistema: una alta variabilidad en el tiempo de la ruta 
(\(\sigma_{LT}\) elevado) exige un alto stock de seguridad para mantener la 
continuidad del servicio.

\subsection{Resiliencia en Cadenas de Suministro}
El enfoque tradicional de eficiencia ha sido complementado por el concepto de
resiliencia, definida como la capacidad de una cadena de suministro para
absorber, adaptarse y recuperarse de disrupciones \cite{Christopher2004}. Este
concepto introduce un compromiso fundamental entre eficiencia (minimización de
costos) y robustez (inversión en redundancia). El sistema de Aysén, como
muestra la Figura~\ref{fig:resilience-tradeoff}, se encuentra en una zona de
aparente eficiencia pero alta vulnerabilidad.

\begin{figure}[htbp]
    \centering
    \begin{tikzpicture}[
        font=\small,
        axis/.style={->, >=Stealth, thick, black!80},
        curve/.style={thick, blue!60, line width=2pt}
    ]
        % Ejes principales
        \draw[axis] (0,0) -- (8,0) node[below] {\textbf{Eficiencia (Costos Bajos)}};
        \draw[axis] (0,0) -- (0,5) node[left, rotate=90, anchor=south] {\textbf{Resiliencia}};
        
        % Curva frontera más suave
        \draw[curve] (1,4.2) .. controls (3,2.8) and (5,1.5) .. (7.5,0.6);
        
        % Áreas de las zonas (más sutiles)
        \fill[blue!8] (0,0) -- (1,4.2) .. controls (3,2.8) and (5,1.5) .. (7.5,0.6) -- (7.5,0) -- cycle;
        \fill[red!8] (1,4.2) .. controls (3,2.8) and (5,1.5) .. (7.5,0.6) -- (7.5,5) -- (0,5) -- (0,4.2) -- cycle;
        
        % Redibujar la curva
        \draw[curve] (1,4.2) .. controls (3,2.8) and (5,1.5) .. (7.5,0.6);
        
        % Etiquetas de las zonas (más simples)
        \node[align=center, font=\footnotesize] at (1.8,3.5) {
            \textbf{Zona de Alta Robustez}\\
            \textcolor{gray}{(Alto costo, alta redundancia)}
        };
        
        \node[align=center, font=\footnotesize] at (5.5,3.2) {
            \textbf{Zona de Alta Vulnerabilidad}\\
            \textcolor{gray}{(Bajo costo, baja robustez)}
        };
        
        % Posición de Aysén (más simple)
        \fill[red] (6,1.2) circle (3pt);
        \node[below right, red, font=\footnotesize] at (6,1.2) {Sistema Aysén};
        
        % Líneas de referencia discretas
        \draw[gray!40, dashed, very thin] (6,0) -- (6,1.2);
        \draw[gray!40, dashed, very thin] (0,1.2) -- (6,1.2);
        
        % Etiquetas en los extremos de los ejes
        \node[below, gray] at (1,0) {Mayor};
        \node[below, gray] at (7,0) {Menor};
        \node[left, gray] at (0,0.5) {Menor};
        \node[left, gray] at (0,4) {Mayor};
        
    \end{tikzpicture}
    \caption{El \textit{trade-off} entre eficiencia y resiliencia en la gestión de cadenas de suministro.}
    \label{fig:resilience-tradeoff}
\end{figure}

\subsection{Dinámicas de Competencia y Coordinación: El Aporte de la Teoría de Juegos}
La estructura de mercado oligopólica del GLP en Aysén introduce una capa de 
complejidad estratégica. La teoría de juegos, y en particular el modelo del 
``Dilema del Prisionero'', ofrece un marco formal para entender por qué la 
acción racional individual (minimizar costos de inventario) puede conducir 
a un resultado colectivo subóptimo (baja resiliencia sistémica). La 
evidencia de que un actor opera con una capacidad ``demasiado reducida para 
su nivel de ventas''~\cite{CIEP2025} sugiere que esta dinámica no es solo 
teórica, sino una práctica observable en el sistema bajo estudio.

\section{Metodología del Dominio de la Solución: La Simulación de Sistemas}
\label{sec:simulation-theory}

La simulación es la imitación del funcionamiento de un sistema del mundo 
real a lo largo del tiempo \cite{Law2015}. Para un sistema estocástico y 
dinámico como el de Aysén, es la única metodología viable para evaluar el 
impacto de diferentes políticas bajo incertidumbre.

\subsection{Simulación de Eventos Discretos (SED)}
La SED es un método de modelado que representa un sistema como una secuencia
cronológica de eventos \cite{Banks2010}. Los componentes fundamentales de
nuestro modelo SED son:
\begin{description}
    \item[Entidades] \texttt{CamiónSuministro} (con atributos: capacidad, origen).
    \item[Recursos] \texttt{PlantaAlmacenamiento} (con atributos: capacidad\_max, 
    nivel\_inventario, ROP), \texttt{RutaTerrestre} (con estado: abierta/cerrada).
    \item[Procesos] \texttt{GeneracionPedidos} (cuando \(I(t) \le R\)), 
    \texttt{ViajeSuministro} (proceso estocástico con duración \(LT\)), 
    \texttt{ConsumoDiario} (reduce \(I(t)\)), \texttt{GeneradorDisrupciones} (cambia el 
    estado de la \texttt{RutaTerrestre}).
\end{description}

\subsection{Protocolos de Credibilidad: Verificación y Validación (V\&V)}
La credibilidad de un estudio de simulación depende críticamente de un 
proceso formal de V\&V.
\begin{itemize}
    \item \textbf{Verificación:} \textit{¿Se construyó el modelo correctamente?} 
    Se asegura que el código implementa fielmente el modelo conceptual.
    \item \textbf{Validación:} \textit{¿Se construyó el modelo correcto?} 
    Se determina si el modelo es una representación suficientemente precisa 
    del sistema real, comparando sus salidas con datos históricos y juicio 
    de expertos.
\end{itemize}

\subsection{Análisis de Resultados: Diseño de Experimentos (DoE)}
Una vez validado, el modelo se usa para la experimentación. El DoE es el 
marco formal para variar sistemáticamente los factores de entrada (ej. 
capacidad de almacenamiento) para cuantificar su efecto en las métricas de 
respuesta (ej. Nivel de Servicio \(\alpha\)). El análisis estadístico de 
estos resultados (ej. ANOVA) permite identificar los parámetros más 
influyentes y probar hipótesis de manera rigurosa.