\chapter{Diseño Metodológico}
\label{chap:metodologia}

Este capítulo detalla el plan de trabajo y los métodos que se emplearán para
alcanzar los objetivos de la investigación. El enfoque se enmarca en las
disciplinas de la Ciencia Computacional y la Ingeniería de Sistemas, adoptando
un proceso sistemático para el diseño, implementación y evaluación del modelo
de simulación. El propósito es establecer un marco de trabajo riguroso,
transparente y replicable.

\section{Fase 1: Modelado Conceptual del Sistema (Objetivo 1)}
\label{sec:conceptual-modeling}

El objetivo de esta fase es la abstracción formal del sistema real en un 
modelo conceptual que sirva como especificación para la implementación.

% --- SECCIÓN REESCRITA PARA MÁXIMA CLARIDAD ---
\subsection{Definición de Límites y Supuestos del Modelo}
\label{sec:limites-supuestos}

Para abordar el problema regional de manera cuantitativa, es imperativo 
establecer un límite de sistema (\textit{system boundary}) claro para el 
modelo computacional. Si bien el impacto de una disrupción en el suministro 
de GLP es de alcance regional, el centro de gravedad logístico y el punto 
de control más crítico de la cadena se encuentra en el nodo de 
almacenamiento primario de Coyhaique.

Por consiguiente, el modelo se centrará en la dinámica de inventarios de 
las tres plantas mayoristas ubicadas en dicho nodo. Se modelará el flujo de 
suministro desde los puntos de origen (Neuquén y Cabo Negro) hasta estas 
plantas, y la salida de producto se modelará como una demanda agregada que 
representa el consumo de la zona de influencia directa.

Este enfoque se justifica porque la resiliencia del nodo de Coyhaique actúa 
como un \textbf{proxy directo} de la resiliencia de toda la región. Un 
quiebre de stock en este punto neurálgico implica, por definición, la 
incapacidad de abastecer a las demás localidades. La modelización de la red 
de distribución de última milla y las dinámicas de inventario en otros puntos 
de la región quedan fuera del alcance de este estudio y se plantean como 
líneas de trabajo futuro.

\section{Enfoque de la Investigación}
\label{sec:research-approach}

Esta investigación adopta un enfoque cuantitativo basado en simulación como
método de experimentación computacional. Dada la naturaleza estocástica y
compleja de la cadena de suministro de GLP, los modelos puramente analíticos
son insuficientes. La simulación permite probar hipótesis y evaluar políticas
bajo condiciones controladas, algo inviable en el sistema real. El desarrollo
se estructura en las tres fases ilustradas en la
Figura~\ref{fig:conceptual-diagram-detailed}.

\begin{figure}[htbp]
    \centering
    \begin{tikzpicture}[
        node distance=4cm,
        box/.style = {rectangle, draw, thick, rounded corners=2pt, 
                     minimum height=1.5cm, text width=2.5cm, 
                     text centered, font=\small, align=center},
        arrow/.style = {->, >=stealth, thick}
    ]
        
        % Componentes principales en línea
        \node[box, fill=blue!15] (A) at (0,0) {Generación de Pedidos};
        \node[box, fill=green!15] (B) at (4.5,0) {Transporte y Logística};
        \node[box, fill=purple!15] (C) at (9,0) {Almacenamiento y Distribución};
        \node[box, fill=orange!15] (D) at (13.5,0) {Consumo Final};
        
        % Flujo principal
        \draw[arrow, blue] (A) -- (B);
        \draw[arrow, blue] (B) -- (C);
        \draw[arrow, blue] (C) -- (D);
        
        % Elemento disruptivo
        \node[box, fill=red!15] (E) at (2.25,-3) {Sistema de Disrupciones};
        \draw[arrow, red, thick] (E) -- (A) node[midway, left, font=\footnotesize, text=red] {Interrumpe};
        \draw[arrow, red, thick] (E) -- (B) node[midway, below, font=\footnotesize, text=red] {Retrasa};
        
        % Parámetros
        \node[box, fill=gray!15] (F) at (6.75,-3) {Base de Parámetros};
        \draw[arrow, dashed, gray] (F) -- (A);
        \draw[arrow, dashed, gray] (F) -- (C);
        \draw[arrow, dashed, gray] (F) -- (D);
        
        % Leyenda simple
        \node[below=1.2cm of F, font=\footnotesize] {
            \textcolor{blue}{\rule{1cm}{1.5pt}} Flujo Principal \quad
            \textcolor{red}{\rule{1cm}{1.5pt}} Disrupciones \quad
            \textcolor{gray}{\rule[0.5pt]{1cm}{1pt}} Parámetros
        };
        
    \end{tikzpicture}
    \caption{Modelo Conceptual del Sistema de Distribución de GLP.}
    \label{fig:conceptual-diagram-detailed}
\end{figure}

\subsection{Parametrización y Modelado Estocástico}

Los parámetros del modelo se dividen en dos categorías: deterministas,
tomados directamente del informe \cite{CIEP2025}, y estocásticos, modelados
mediante distribuciones de probabilidad calibradas con datos históricos.

\subsubsection{Parámetros Deterministas}

\begin{itemize}
    \item \textbf{Capacidad de Almacenamiento:}
    \begin{itemize}
        \item Status Quo: \SI{431}{TM} (Abastible: 150, Lipigas: 240, Gasco: 41)
        \item Propuesta 10.4: \SI{681}{TM} (incremento de \SI{250}{TM})
    \end{itemize}

    \item \textbf{Política de Inventario $(Q, R)$:}
    \begin{itemize}
        \item Punto de Reorden ($R$): 50\% de la capacidad
        \item Cantidad de Pedido ($Q$): 50\% de la capacidad
        \item Inventario Inicial: 60\% de la capacidad (arranque realista)
    \end{itemize}

    \item \textbf{Lead Time Nominal:} \SI{6}{días} (tiempo promedio de entrega
    desde Cabo Negro o Neuquén hasta Coyhaique)

    \item \textbf{Horizonte de Simulación:} \SI{365}{días} (1 año)
\end{itemize}

\subsubsection{Variables Estocásticas y Distribuciones}

Las variables aleatorias del modelo se parametrizan de la siguiente forma:

\begin{enumerate}
    \item \textbf{Frecuencia de Disrupciones:} Se modela como un proceso de
    Poisson con tasa $\lambda = 4$ eventos/año. El tiempo entre disrupciones
    consecutivas sigue una distribución \textbf{Exponencial}:
    \[ T_{\text{entre}} \sim \text{Exp}\left(\frac{\lambda}{365}\right) = \text{Exp}(0.0110) \]
    donde $\lambda$ corresponde a la frecuencia de Nivel 4 identificada en la
    matriz de riesgos~\cite{CIEP2025}.

    \item \textbf{Duración de Disrupciones:} Se emplea una distribución
    \textbf{Triangular}($a$, $b$, $c$) calibrada con datos históricos:
    \[ D_{\text{disrup}} \sim \text{Triangular}(a, b, c) \]
    Los parámetros varían según el escenario experimental:
    \begin{itemize}
        \item Corta: $a = 3$, $b = 3.5$, $c = 7$ días
        \item Media: $a = 3$, $b = 7$, $c = 14$ días
        \item Larga: $a = 3$, $b = 10.5$, $c = 21$ días (conflicto Argentina 2021)
    \end{itemize}
    La distribución triangular permite modelar la incertidumbre con parámetros
    interpretables: mínimo histórico, valor más probable, y máximo observado.

    \item \textbf{Demanda Diaria:} Se modela como un proceso estocástico con
    componente estacional y ruido:
    \[ D(t) = D_{\text{base}} \cdot \left(1 + 0.25 \sin\left(\frac{2\pi(t - 172)}{365}\right)\right) \cdot \epsilon(t) \]
    donde:
    \begin{itemize}
        \item $D_{\text{base}} = \SI{52.5}{TM/día}$ (demanda promedio del mes
        de mayor consumo)
        \item El término sinusoidal modela la estacionalidad invernal
        (pico en julio, día $\approx 200$)
        \item $\epsilon(t) \sim \mathcal{N}(1.0, 0.15)$ es el ruido estocástico
        diario ($\pm 15\%$ de variabilidad)
    \end{itemize}
    La demanda base de \SI{52.5}{TM/día} se calibró para representar el escenario
    de estrés del sistema (mes de mayor consumo), lo que genera una autonomía
    conservadora de $\approx \SI{5}{días}$ en el escenario Status Quo, frente a
    los \SI{8.2}{días} calculados con demanda promedio anual.
\end{enumerate}

\subsubsection{Generación de Números Aleatorios}

Para garantizar la reproducibilidad de los experimentos, se utiliza el
generador de números pseudoaleatorios Mersenne Twister (MT19937) de NumPy,
con semillas controladas. Cada réplica $r$ de la configuración $c$ emplea
una semilla única:
\[ s_{c,r} = s_{\text{base}} + (c - 1) \times 1000 + r \]
donde $s_{\text{base}} = 42$. Esta estrategia asegura independencia
estadística entre réplicas y reproducibilidad exacta de los resultados.

\section{Fase 2: Implementación del Prototipo (Objetivo 2)}
\label{sec:implementation}

\subsection{Stack Tecnológico y Arquitectura de Software}
La selección de herramientas busca maximizar la flexibilidad y la 
reproducibilidad científica:

\begin{itemize}
    \item \textbf{Núcleo:} \textbf{Python 3.x} y \textbf{SimPy}, por su 
    capacidad para modelar procesos complejos de forma nativa.
    
    \item \textbf{Análisis y Visualización:} \textbf{Pandas}, \textbf{Matplotlib}, 
    y \textbf{Seaborn} para el procesamiento de datos y la generación de 
    gráficos.
    
    \item \textbf{Reproducibilidad:} Se utilizará \textbf{Git} para el control 
    de versiones y un gestor de entornos virtuales (\textbf{venv} o 
    \textbf{Conda}) para encapsular las dependencias.
\end{itemize}

La arquitectura del software seguirá un patrón \textbf{Modelo-Experimento-Análisis}, 
separando la lógica de la simulación (\texttt{modelo.py}), de la orquestación de 
escenarios (\texttt{experimento.py}) y de los scripts de análisis (\texttt{analisis.py}). 
Los parámetros se externalizarán a un archivo \texttt{config.json}.

\section{Fase 3: Evaluación y Experimentación (Objetivo 3)}
\label{sec:validation-experimentation}

\subsection{Protocolo de Verificación y Validación (V\&V)}
Se aplicará un protocolo formal para establecer la credibilidad del modelo:

\begin{itemize}
    \item \textbf{Verificación:} Se realizarán \textit{code walkthroughs} y 
    pruebas de componentes deterministas para asegurar que el código refleja 
    el modelo conceptual.
    
    \item \textbf{Validación:} Se empleará un enfoque de múltiples facetas:
    \begin{itemize}
        \item \textbf{Validación de Datos Históricos:} Se compararán las 
        distribuciones estadísticas (media, varianza) de las métricas 
        clave del modelo (ej. días de autonomía) con los valores de 
        referencia del informe \cite{CIEP2025} (ej. media de 8.2 días).
        
        \item \textbf{Validación por Juicio de Expertos (Face Validity):} 
        Se realizarán ``pruebas de Turing'' para modelos, donde se 
        presentarán trazas de salida del modelo a los expertos técnicos 
        de la SEC para que evalúen su plausibilidad operativa.
    \end{itemize}
\end{itemize}

\subsection{Diseño de Experimentos (DoE)}
Para probar la hipótesis central, se ejecutará un \textbf{Diseño Factorial \(2 \times 3\)}. 
Se realizarán múltiples replicaciones para cada combinación de factores para 
capturar la variabilidad estocástica.

\begin{table}[htbp]
    \centering
    \caption{Diseño Experimental para la Evaluación de la Resiliencia.}
    \label{tab:doe}
    \begin{tabular}{@{}ll@{}}
        \toprule
        \textbf{Componente} & \textbf{Especificación} \\ \midrule
        \textbf{Factores} & 1. Capacidad de Almacenamiento (Endógeno) \\
                          & 2. Duración de Disrupción (Exógeno) \\
        \addlinespace
        \textbf{Niveles Factor 1} & Nivel 1: Status Quo (431 ton) \\
                                 & Nivel 2: Propuesta 10.4 (681 ton) \\
        \addlinespace
        \textbf{Niveles Factor 2} & Nivel 1: Corta (7 días) \\
                                 & Nivel 2: Media (14 días) \\
                                 & Nivel 3: Larga (21 días) \\
        \addlinespace
        \textbf{Variables de Respuesta} & 1. Nivel de Servicio (\%) \\
        (KPIs)                          & 2. Probabilidad de Quiebre de Stock \\ \bottomrule
    \end{tabular}
\end{table}

Los resultados se analizarán mediante un \textbf{Análisis de Varianza (ANOVA)} 
para determinar la significancia estadística de los efectos principales de 
cada factor y de sus interacciones, proveyendo así la evidencia para 
confirmar o refutar la hipótesis.