\begin{figure}[htbp]
    \centering
    \begin{tikzpicture}[
        font=\footnotesize,
        process/.style={
            rectangle, draw=black, line width=1pt,
            minimum height=8cm, minimum width=2.5cm,
            text centered, font=\small, align=center
        },
        step/.style={
            rectangle, draw=black!40, line width=0.5pt,
            minimum height=0.7cm, minimum width=2cm,
            text centered, font=\scriptsize, align=center
        },
        resource/.style={
            rectangle, draw=black, line width=1pt,
            minimum height=2.5cm, minimum width=2.2cm,
            text centered, font=\small, align=center,
            fill=gray!15
        },
        dashed_arrow/.style={->, >=stealth, line width=0.8pt, dashed}
    ]

        % Proceso 1
        \node[process, fill=gray!5] (p1) at (0,0) {
            \textbf{Proceso 1}\\
            \textbf{Demanda}
        };

        \node[step] at (0,3) {Calcular $D(t)$};
        \node[step] at (0,2) {Despachar};
        \node[step] at (0,1) {Registrar};
        \node[step] at (0,0) {Esperar 1 día};
        \node[step] at (0,-1) {$\cdots$};

        % Proceso 2
        \node[process, fill=gray!5] (p2) at (3.5,0) {
            \textbf{Proceso 2}\\
            \textbf{Reabastecimiento}
        };

        \node[step] at (3.5,3) {Revisar $I$};
        \node[step] at (3.5,2) {$I \leq R$?};
        \node[step] at (3.5,1) {Crear Pedido};
        \node[step] at (3.5,0) {Esperar 1 día};
        \node[step] at (3.5,-1) {$\cdots$};

        % Proceso 3
        \node[process, fill=gray!5] (p3) at (7,0) {
            \textbf{Proceso 3}\\
            \textbf{Disrupciones}
        };

        \node[step] at (7,3) {Generar $T$};
        \node[step] at (7,2) {Esperar $T$};
        \node[step] at (7,1) {Bloquear Ruta};
        \node[step] at (7,0) {Esperar $D$};
        \node[step] at (7,-1) {$\cdots$};

        % Recursos compartidos
        \node[resource] (hub) at (3.5,-5.5) {
            \textbf{Hub}\\[0.15cm]
            Inventario $I$
        };

        \node[resource] (ruta) at (7,-5.5) {
            \textbf{Ruta}\\[0.15cm]
            Estado
        };

        % Interacciones
        \draw[dashed_arrow] (0,-2) -- (hub.north west)
            node[midway, above, sloped, font=\tiny] {consume};

        \draw[dashed_arrow] (3.5,-2) -- (hub.north)
            node[midway, right, font=\tiny] {lee/escribe};

        \draw[dashed_arrow] (4,-2) -- (ruta.north west)
            node[midway, above, sloped, font=\tiny] {consulta};

        \draw[dashed_arrow] (7,-2) -- (ruta.north)
            node[midway, right, font=\tiny] {modifica};

    \end{tikzpicture}
    \caption{Diagrama de interacción de los tres procesos concurrentes del simulador.}
    \label{fig:procesos-concurrentes}
\end{figure}
