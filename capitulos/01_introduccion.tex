\chapter{Introducción}
\label{chap:introduccion}

La gestión de cadenas de suministro para recursos energéticos en geografías 
aisladas y complejas constituye un problema de optimización estocástica de 
alta relevancia. La inherente variabilidad de las condiciones operacionales, 
sumada a la criticidad de asegurar la continuidad del servicio, demanda el 
desarrollo de herramientas analíticas que trasciendan los enfoques 
determinísticos tradicionales. La Región de Aysén, en la Patagonia chilena, 
representa un caso de estudio paradigmático de esta problemática, 
particularmente en lo que respecta a su sistema de suministro de Gas Licuado 
de Petróleo (GLP).

El presente trabajo de tesis aborda la vulnerabilidad estructural de dicha 
cadena de suministro. El objetivo fundamental es el diseño y la validación 
de un artefacto computacional ---un prototipo de simulación de eventos 
discretos--- que permita el análisis cuantitativo de la resiliencia del 
sistema. Este laboratorio virtual se centrará en el nodo logístico de 
Coyhaique, el corazón del sistema de distribución regional, y se empleará 
para modelar la interacción de parámetros logísticos clave, cuantificar el 
impacto de escenarios de disrupción y, en última instancia, proveer una base 
empírica para la evaluación de estrategias de mitigación.

La investigación se estructura de manera deductiva para guiar al lector desde 
el contexto general hasta los detalles técnicos y los resultados. El 
\cref{chap:planteamiento-problema} contextualiza y justifica la relevancia 
del problema. El \cref{chap:hipotesis} formaliza la pregunta de 
investigación y la hipótesis central que guían el estudio. El 
\cref{chap:objetivos} delimita las metas del proyecto de manera precisa. Los 
capítulos \ref{chap:marco-teorico} y \ref{chap:estado-del-arte} establecen 
los fundamentos teóricos y el estado del arte en la materia. Finalmente, el 
\cref{chap:metodologia} detalla el diseño metodológico para el desarrollo y 
la validación del prototipo.