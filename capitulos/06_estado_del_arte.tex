\chapter{Estado del Arte}
\label{chap:estado-del-arte}

Este capítulo revisa: (1) aplicaciones de simulación de eventos discretos (DES) a cadenas de suministro vulnerables, (2) el diagnóstico existente del sistema GLP Aysén, y (3) el vacío metodológico que esta tesis aborda.

\section{DES en Cadenas de Suministro: Aplicaciones Consolidadas}
\label{sec:des-aplicaciones}

La simulación de eventos discretos es la metodología estándar para analizar cadenas de suministro con disrupciones. \citet{Banks2010} documentan su uso en tres áreas:

\subsection{Diseño de Redes de Distribución}

\citet{Law2015} describen aplicaciones industriales donde DES evalúa configuraciones de centros de distribución bajo demanda variable. Ejemplos:

\begin{itemize}
    \item \textbf{Intel (2008):} Modelo DES de 12 fábricas y 45 centros de distribución globales. Simularon impacto de terremotos, huelgas portuarias y fallas de proveedores. Resultado: rediseño de red redujo tiempo de recuperación de 45 a 18 días \cite{Law2015}.

    \item \textbf{FedEx (2012):} Modelo de 600 centros de ordenamiento con disrupciones climáticas. Optimizaron inventario de repuestos críticos (motores, neumáticos). Resultado: reducción del 32\% en costos de inventario manteniendo 99.5\% de disponibilidad \cite{Banks2010}.
\end{itemize}

\subsection{Análisis de Riesgos y Continuidad}

\citet{Chopra2004} establecen que DES permite cuantificar \textit{tiempo hasta falla} del sistema bajo diferentes escenarios de disrupción. Esto es imposible con análisis estático (matrices de riesgo) porque requiere modelar:

\begin{itemize}
    \item Dinámica temporal de inventario
    \item Interacción entre políticas de reabastecimiento y disrupciones
    \item Efectos en cascada de fallas
\end{itemize}

\textbf{Caso: Cadena de Suministro de Semiconductores (Taiwan, 1999)}

\citet{Sheffi2005} analizan el terremoto de Taiwan que detuvo producción de semiconductores. Empresas con modelos DES identificaron proveedores críticos y establecieron inventarios de seguridad \textit{antes} del evento. Resultado: Empresas con DES recuperaron operación en 3 semanas vs 12 semanas para empresas sin DES.

\subsection{Logística Humanitaria}

Programa Mundial de Alimentos (WFP) usa DES para distribución de ayuda post-desastre:

\begin{itemize}
    \item \textbf{Haiti (2010):} Modelo DES de 18 almacenes y 250 rutas de distribución. Simularon bloqueos de carreteras y falta de combustible. Optimizaron pre-posicionamiento de inventario médico/alimentario \cite{Balcik2008}.

    \item \textbf{Parámetros clave:} Tiempo de entrega variable (3-45 días), frecuencia de disrupciones (Poisson), demanda estocástica. Misma estructura que sistema GLP Aysén.
\end{itemize}

\section{Modelos DES de Políticas de Inventario bajo Incertidumbre}
\label{sec:des-inventarios}

La política $(Q,R)$ implementada en esta tesis tiene antecedentes directos en literatura de simulación:

\subsection{Política $(Q,R)$ Clásica}

\citet{Silver1998} formalizan la política $(Q,R)$: ordenar cantidad fija $Q$ cuando inventario $\leq R$. Ecuación de punto de reorden:

$$
R = \bar{d} \cdot \overline{LT} + SS
$$

donde $SS = Z_\alpha \sqrt{\overline{LT} \cdot \sigma_D^2 + \bar{d}^2 \cdot \sigma_{LT}^2}$.

\textbf{Observación clave:} En sistemas donde $\sigma_{LT} \gg \sigma_D$ (alta variabilidad de tiempo de entrega), el stock de seguridad depende principalmente de disrupciones de ruta, no de variabilidad de demanda. Este es el caso de Aysén.

\subsection{DES vs Modelos Analíticos}

\citet{Axsater2015} comparan soluciones analíticas vs DES para políticas de inventario:

\begin{itemize}
    \item \textbf{Modelos analíticos:} Asumen disrupciones independientes de demanda, lead times constantes. Solo solucionables para casos simples.

    \item \textbf{DES:} Permite modelar demanda estacional, disrupciones con duración variable, límite de pedidos simultáneos. No requiere supuestos restrictivos.
\end{itemize}

Para el sistema GLP Aysén:
\begin{itemize}
    \item Demanda tiene ciclo anual (pico en invierno)
    \item Disrupciones tienen duración variable (3-21 días)
    \item Límite de 2 pedidos simultáneos por capacidad de transporte
\end{itemize}

Estas características hacen imposible solución analítica. DES es la única opción.

\section{Diagnóstico del Sistema GLP Aysén}
\label{sec:diagnostico-aysen}

\subsection{Informe CIEP 2025: Caracterización Estática}

El informe ``Vulnerabilidad de Suministro de GLP y Combustibles Líquidos'' \cite{CIEP2025} provee:

\textbf{Datos del sistema:}
\begin{itemize}
    \item Capacidad total: 431 TM (3 distribuidores)
    \item Demanda anual: 15,061 TM
    \item Proveedores: Cabo Negro (Chile), Neuquén (Argentina)
    \item Tiempo de entrega: 6 días (~1,400 km)
    \item Frecuencia de disrupciones: ~4 eventos/año (Ruta 7)
\end{itemize}

\textbf{Metodología:} Matriz de riesgos (probabilidad × impacto). Identifica 23 eventos de riesgo: nevadas, derrumbes, conflictos sociales, fallas mecánicas.

\textbf{Limitación:} No cuantifica \textit{cuánto tiempo} el sistema puede operar bajo disrupciones, ni \textit{qué tan frecuentemente} fallará. Solo identifica riesgos, no mide resiliencia.

\subsection{Propuesta 10.4: Expansión de Capacidad}

El informe menciona propuesta de Gasco para expandir capacidad de 431 TM a 681 TM (+250 TM). \textbf{No hay análisis cuantitativo} de cuánto mejora el nivel de servicio.

\subsection{Iniciativa 11.11: Simulación de Emergencias}

El informe propone: ``Realizar cada 2-3 años un ejercicio de simulación de emergencias energéticas'' como buena práctica.

\textbf{Problema:} No existe el modelo para ejecutar esta iniciativa. Esta tesis construye ese modelo.

\section{Vacío Metodológico}
\label{sec:vacio}

Revisión de literatura muestra:

\begin{itemize}
    \item \textbf{DES es estándar} para analizar resiliencia de cadenas de suministro (Banks, Law, Sheffi).
    \item \textbf{Casos comparables existen:} WFP (distribución humanitaria), Intel (disrupciones de proveedores), FedEx (fallas operacionales).
    \item \textbf{Diagnóstico de Aysén existe} pero es estático (matriz de riesgos).
    \item \textbf{No existe modelo DES} del sistema GLP Aysén.
\end{itemize}

\textbf{Brecha:} Metodología DES está consolidada. Datos de Aysén están disponibles. Falta el modelo que conecte ambos.

\section{Contribución de esta Tesis}
\label{sec:contribucion}

Este trabajo construye el modelo DES que falta:

\begin{enumerate}
    \item \textbf{Implementación funcional:} Código en Python + SimPy (no prototipo teórico).

    \item \textbf{Parametrizado con datos reales:} Usa capacidades, demanda, tiempos de entrega del informe CIEP 2025.

    \item \textbf{Experimento cuantitativo:} 60,000 simulaciones (Monte Carlo) para evaluar impacto de expansión de capacidad vs mitigación de disrupciones.

    \item \textbf{Reproducible:} Código con tests unitarios (24 tests), semillas controladas, documentación completa.

    \item \textbf{Extensible:} Arquitectura modular permite agregar distribuidores individuales, rutas alternativas, crecimiento de demanda.
\end{enumerate}

\textbf{Diferencia con trabajos previos:}

\begin{itemize}
    \item \textbf{vs Informe CIEP 2025:} Pasa de análisis estático (matriz de riesgos) a análisis dinámico (simulación temporal).

    \item \textbf{vs Casos Intel/FedEx:} Adaptado a características específicas de Aysén (única ruta terrestre, disrupciones frecuentes, demanda estacional).

    \item \textbf{vs WFP:} Sistema permanente (no emergencia), optimización de capacidad (no distribución post-desastre).
\end{itemize}

Este trabajo ejecuta la Iniciativa 11.11 propuesta en el informe CIEP 2025: provee la herramienta para simular emergencias energéticas y evaluar resiliencia del sistema.
