\chapter{Discusión}
\label{chap:discusion}

Este capítulo interpreta los resultados presentados en el \cref{chap:resultados},
situándolos en el contexto más amplio de la teoría de resiliencia de cadenas
de suministro y sus implicaciones para la política pública energética en la
Región de Aysén. Se estructura en cuatro secciones: primero, la interpretación
teórica de los hallazgos; segundo, las implicaciones prácticas para la toma
de decisiones; tercero, las limitaciones del estudio; y cuarto, las
oportunidades para investigación futura.

\section{Interpretación Teórica de los Hallazgos}
\label{sec:interpretacion-teorica}

\subsection{La Dominancia de los Factores Exógenos}

El hallazgo central de esta investigación ---que la resiliencia es 2,07 veces
más sensible a la duración de disrupciones que a la capacidad de
almacenamiento--- valida y cuantifica una intuición que ha estado presente en
el diagnóstico cualitativo del sistema \cite{CIEP2025}, pero que hasta ahora
carecía de evidencia empírica rigurosa.

Este resultado es consistente con la teoría de resiliencia de cadenas de
suministro propuesta por Christopher y Peck \cite{Christopher2004}, que
establece una distinción fundamental entre:

\begin{itemize}
    \item \textbf{Robustez interna} (capacidad de absorber disrupciones
    mediante redundancia de inventarios), y
    \item \textbf{Robustez externa} (capacidad de evitar o acortar las
    disrupciones mediante redundancia de rutas o respuesta rápida).
\end{itemize}

El sistema de Aysén exhibe un desequilibrio marcado: posee una robustez
interna limitada pero ajustada para disrupciones de corta duración, pero
carece por completo de robustez externa, ya que depende de una ruta única.
Los resultados del experimento demuestran que, en este régimen de
vulnerabilidad, invertir en robustez interna (más almacenamiento) ofrece
rendimientos decrecientes, mientras que la exposición a disrupciones externas
domina el comportamiento del sistema.

\subsection{El Efecto No Lineal de las Disrupciones Largas}

Un hallazgo particularmente relevante es que el impacto de las disrupciones
largas (21 días) no es simplemente proporcional a su duración, sino que genera
un efecto en cascada. El nivel de servicio en el escenario Status Quo-Larga
(97,70\%) cae significativamente más que lo que se esperaría de una relación
lineal entre duración de disrupción y degradación de servicio.

Este comportamiento se explica por la dinámica de la política de inventario
$(Q,R)$: cuando una disrupción supera el tiempo de cobertura combinado del
inventario disponible y el stock de seguridad, el sistema entra en un régimen
de \textit{quiebre sostenido}, donde cada día de disrupción adicional se
traduce directamente en demanda insatisfecha. Este umbral no lineal es
característico de sistemas de inventario bajo incertidumbre en el lead time.

En el caso de Aysén, con una autonomía promedio de aproximadamente 5 días y
disrupciones que pueden extenderse hasta 21 días, el sistema opera
peligrosamente cerca de este umbral de colapso, lo que explica la alta
sensibilidad observada.

\section{Implicaciones para la Política Pública}
\label{sec:implicaciones-practicas}

\subsection{Evaluación Costo-Beneficio de Inversiones}

Los resultados del experimento permiten realizar una evaluación comparativa
de dos estrategias de inversión para mejorar la resiliencia del sistema:

\textbf{Opción A: Expansión de Capacidad de Almacenamiento}

Inversión en la construcción de nueva infraestructura de almacenamiento para
aumentar la capacidad de 431 TM a 681 TM (+250 TM, propuesta Gasco 10.4).

\begin{itemize}
    \item \textbf{Costo estimado:} Según el informe \cite{CIEP2025}, la
    inversión en infraestructura de almacenamiento de GLP tiene un costo
    aproximado de USD 6.000 por tonelada de capacidad instalada. Una expansión
    de 250 TM implicaría una inversión cercana a \textbf{USD 1,5 millones}.

    \item \textbf{Beneficio en resiliencia:} Mejora del nivel de servicio de
    0,69 puntos porcentuales (de 98,82\% a 99,50\%).

    \item \textbf{Costo por punto de mejora:} Aproximadamente USD 2,2 millones
    por punto porcentual de nivel de servicio.
\end{itemize}

\textbf{Opción B: Mitigación de Disrupciones}

Inversión en medidas que reduzcan la duración o frecuencia de las disrupciones.
Ejemplos: mejora de infraestructura vial (puentes con mayor capacidad de carga
para permitir camiones de 45 toneladas), protocolos de emergencia con
Argentina, sistema de alerta temprana, rutas alternativas.

\begin{itemize}
    \item \textbf{Beneficio potencial en resiliencia:} Si se logra reducir la
    duración máxima de disrupciones de 21 días a 14 días, el nivel de servicio
    mejoraría en aproximadamente 0,7 puntos porcentuales (interpolando entre
    los escenarios Media y Larga).

    \item \textbf{Retorno relativo:} Por cada punto porcentual de mejora en
    resiliencia, la mitigación de disrupciones es \textbf{2,07 veces más
    efectiva} que la expansión de capacidad.
\end{itemize}

\textbf{Recomendación:} Dada la alta sensibilidad del sistema a factores
exógenos, se recomienda priorizar inversiones en mitigación de disrupciones
por sobre la expansión de capacidad de almacenamiento. Esto no implica que
la capacidad adicional sea innecesaria, sino que debería ser complementada
---y no sustituir--- inversiones en redundancia de rutas y gestión de
disrupciones.

\subsection{Estrategias de Mitigación Específicas}

Basándose en los resultados, se proponen las siguientes estrategias de
mitigación priorizadas:

\begin{enumerate}
    \item \textbf{Diversificación de rutas de suministro:} Habilitar el Paso
    Río Jeinimeni como ruta alternativa de emergencia. Aunque implica un mayor
    costo logístico, permitiría mantener el flujo de suministro durante cierres
    del Paso Huemules.

    \item \textbf{Protocolos de emergencia binacionales:} Establecer convenios
    formales con autoridades argentinas para priorizar el tránsito de camiones
    cisterna durante conflictos sociales o cierres administrativos.

    \item \textbf{Sistema de alerta temprana:} Implementar un sistema de
    monitoreo en tiempo real de las condiciones climáticas y sociales en la
    ruta, permitiendo anticipar disrupciones y adelantar pedidos.

    \item \textbf{Mejora de infraestructura crítica:} Reforzar los tramos más
    vulnerables de la Ruta 7 (cuesta Queulat, El Diablo) y actualizar puentes
    para permitir el tránsito de camiones de 45 toneladas, duplicando la
    capacidad de carga por viaje.

    \item \textbf{Stock de emergencia coordinado:} Establecer un stock de
    seguridad adicional compartido entre los tres distribuidores, gestionado
    de forma cooperativa durante emergencias.
\end{enumerate}

\subsection{El Rol de la Coordinación entre Actores}

Los resultados también sugieren una implicación menos evidente pero igualmente
importante: la necesidad de coordinación entre los distribuidores. La dinámica
actual del mercado oligopólico, donde cada actor minimiza individualmente su
inventario, genera un óptimo privado que resulta subóptimo a nivel sistémico
---un ejemplo clásico del Dilema del Prisionero aplicado a la gestión de
inventarios.

El modelo demostró que incrementar la capacidad total del sistema en un 58\%
mejora el nivel de servicio, pero la mejora es modesta comparada con el costo
de la inversión. Una estrategia alternativa sería optimizar el uso de la
capacidad existente mediante acuerdos de stock compartido o intercambio de
inventario durante emergencias, lo que podría generar un beneficio similar
con una inversión marginal.

\section{Limitaciones del Estudio}
\label{sec:limitaciones}

Este estudio, como toda investigación científica, tiene limitaciones que deben
ser reconocidas para una interpretación adecuada de sus resultados.

\subsection{Simplificaciones del Modelo}

El modelo de simulación desarrollado es, por diseño, una abstracción
simplificada del sistema real. Las principales simplificaciones son:

\begin{enumerate}
    \item \textbf{Hub agregado:} El modelo representa las tres plantas de
    almacenamiento (Abastible, Lipigas, Gasco) como un único hub con inventario
    agregado. Esta simplificación es válida para analizar la resiliencia a
    nivel sistémico, pero no captura dinámicas competitivas ni quiebres de
    stock diferenciados por distribuidor.

    \item \textbf{Ruta única abstracta:} El modelo no diferencia entre las
    rutas desde Cabo Negro y Neuquén, representándolas como una ruta única
    con parámetros promedio. Esto es adecuado dado que ambas rutas comparten
    el cuello de botella crítico (Paso Huemules), pero no permite evaluar el
    beneficio de diversificar fuentes de aprovisionamiento.

    \item \textbf{Demanda agregada:} El modelo no distingue entre GLP granel
    y envasado, ni modela la red de distribución de última milla a las 10
    comunas de la región. El enfoque se centra en el nodo crítico (Coyhaique),
    que actúa como proxy de la resiliencia regional.

    \item \textbf{Política de inventario fija:} El modelo emplea una política
    $(Q,R)$ con parámetros constantes. En la realidad, los distribuidores
    podrían ajustar dinámicamente sus políticas en respuesta a disrupciones
    previstas (ej. aumentar pedidos antes del invierno).
\end{enumerate}

Estas simplificaciones fueron deliberadas y justificadas en el diseño del
modelo (\cref{chap:metodologia}). No obstante, futuras extensiones del modelo
podrían relajar estos supuestos para analizar escenarios más complejos.

\subsection{Limitaciones de los Datos de Entrada}

Los parámetros del modelo se basaron en datos del informe CNE 2024 y en
estimaciones de los distribuidores. Sin embargo:

\begin{itemize}
    \item La \textbf{frecuencia de disrupciones} se basó en la matriz de
    riesgos del informe, que califica eventos de Nivel 4 con una frecuencia
    esperada de 4 veces/año. Datos históricos más extensos permitirían una
    calibración más precisa.

    \item La \textbf{duración de disrupciones} se parametrizó con una
    distribución triangular basada en el evento extremo de 2021 (conflicto
    Argentina, 21 días). Más datos históricos permitirían ajustar una
    distribución empírica.

    \item La \textbf{autonomía del sistema} en el modelo (5,01 días) es menor
    a la calculada con datos reales (8,2 días). Esto se atribuye a que la
    demanda base se calibró con el mes de mayor consumo. Una calibración más
    fina permitiría ajustar este parámetro.
\end{itemize}

\subsection{Alcance Temporal y Estocástico}

Cada simulación cubrió un horizonte de 365 días con 30 réplicas por
configuración. Si bien esto es suficiente para capturar la variabilidad
estocástica de las disrupciones, un análisis de resiliencia a largo plazo
(ej. 5-10 años) podría revelar patrones adicionales, como el efecto del
crecimiento de la demanda (3,8\% anual) o cambios climáticos que afecten la
frecuencia de nevadas.

\section{Oportunidades para Investigación Futura}
\label{sec:investigacion-futura}

Los resultados de este estudio abren múltiples líneas de investigación futura,
tanto metodológicas como aplicadas:

\subsection{Extensiones del Modelo de Simulación}

\begin{enumerate}
    \item \textbf{Modelo multi-agente:} Desarrollar un modelo basado en
    agentes que represente a cada distribuidor individualmente, permitiendo
    analizar dinámicas competitivas, estrategias de coordinación y el efecto
    de políticas regulatorias.

    \item \textbf{Optimización de políticas de inventario:} Utilizar el
    modelo de simulación como función objetivo en un algoritmo de optimización
    para determinar los parámetros óptimos de la política $(Q,R)$ bajo
    diferentes escenarios de riesgo.

    \item \textbf{Análisis de rutas alternativas:} Extender el modelo para
    evaluar el impacto de habilitar rutas alternativas (ej. Paso Río Jeinimeni)
    o modos de transporte alternativos (ej. barcaza energética).

    \item \textbf{Incorporación de crecimiento de demanda:} Modelar la
    evolución temporal de la demanda con tasa de crecimiento del 3,8\% anual,
    proyectando la resiliencia del sistema a 5 y 10 años.
\end{enumerate}

\subsection{Validación con Datos Operativos}

Una limitación actual es que el modelo no ha sido validado con datos
operativos históricos detallados (series temporales de inventario, pedidos,
disrupciones reales). Establecer un convenio con los distribuidores para
acceder a estos datos permitiría:

\begin{itemize}
    \item Validar la precisión del modelo en predecir quiebres de stock
    reales,
    \item Calibrar las distribuciones de probabilidad de disrupciones con
    datos empíricos, y
    \item Ajustar los parámetros de las políticas de inventario a las prácticas
    reales de cada distribuidor.
\end{itemize}

\subsection{Integración con Sistema de Gestión en Tiempo Real}

El prototipo actual es una herramienta de análisis \textit{ex-ante} (evaluar
políticas antes de implementarlas). Una extensión natural sería integrarlo con
un sistema de información en tiempo real que permita:

\begin{itemize}
    \item Monitorear el nivel de inventario de los distribuidores en tiempo
    real,
    \item Generar alertas tempranas cuando el sistema se aproxime a umbrales
    críticos,
    \item Simular escenarios de emergencia en curso para apoyar la toma de
    decisiones operativas durante crisis.
\end{itemize}

Esta evolución transformaría el modelo de una herramienta de planificación
estratégica a un sistema de soporte a decisiones para gestión de emergencias.

\subsection{Aplicación a Otros Sistemas Energéticos}

La metodología desarrollada en esta tesis no es exclusiva del sistema de GLP
de Aysén. Podría aplicarse a otros sistemas energéticos vulnerables de la
región, como:

\begin{itemize}
    \item El suministro de combustibles líquidos (diésel, gasolina),
    \item La red de distribución eléctrica (análisis de resiliencia ante
    fallas en líneas de transmisión),
    \item El abastecimiento de leña (recurso crítico para calefacción).
\end{itemize}

Cada uno de estos sistemas comparte características con el GLP (dependencia de
rutas críticas, estacionalidad de demanda, exposición a disrupciones
climáticas), lo que hace que el enfoque metodológico sea transferible.

\section{Resumen del Capítulo}

Este capítulo interpretó los resultados experimentales en el contexto de la
teoría de resiliencia de cadenas de suministro y extrajo implicaciones
prácticas para la política pública energética en Aysén. Los hallazgos
principales son:

\begin{enumerate}
    \item La dominancia de los factores exógenos es consistente con la teoría
    de resiliencia y refleja un desequilibrio entre robustez interna y externa.

    \item Las inversiones en mitigación de disrupciones ofrecen un mayor
    retorno en resiliencia que la expansión de capacidad de almacenamiento
    (2,07 veces más efectivas).

    \item Las limitaciones del modelo (simplificaciones deliberadas,
    calibración pendiente) deben considerarse al interpretar los resultados
    absolutos, pero no afectan la validez de la prueba de hipótesis.

    \item Múltiples oportunidades de investigación futura permitirían extender
    este trabajo a modelos más complejos, validación con datos reales, y
    aplicación a otros sistemas energéticos.
\end{enumerate}

El capítulo final presentará las conclusiones generales de la investigación
y recomendaciones específicas para los actores relevantes.
