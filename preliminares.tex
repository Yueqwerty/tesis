% --- Agradecimientos ---
\chapter*{Agradecimientos}
\addcontentsline{toc}{chapter}{Agradecimientos} % Añadir a la tabla de contenidos

Deseo expresar mi agradecimiento a las personas que hicieron posible la 
realización de esta memoria.

A mi familia, por su constante apoyo y paciencia a lo largo de mis años de 
estudio. Su confianza fue fundamental para llegar a esta etapa.

A mi mejor amiga, por su invaluable amistad y por su 
aliento en los momentos más difíciles de este proceso.

A mi mentora, Profesora Natacha Pino Acuña, por su guía, tiempo y 
dedicación. Sus conocimientos y consejos fueron esenciales para el 
desarrollo de esta memoria.

\clearpage

% --- Resumen ---
\chapter*{Resumen}
\addcontentsline{toc}{chapter}{Resumen}

La cadena de suministro de Gas Licuado de Petróleo (GLP) en la Región de Aysén 
constituye un sistema sociotécnico de alta criticidad, caracterizado por una 
topología logística sin redundancia y una capacidad de respuesta endógena 
insuficiente para mitigar las disrupciones exógenas recurrentes. El 
diagnóstico técnico actual, si bien es exhaustivo, es de naturaleza estática 
y carece de herramientas para evaluar dinámicamente el impacto de los riesgos 
o el retorno en resiliencia de las inversiones propuestas.

Este trabajo de titulación aborda dicha brecha metodológica mediante el 
diseño, implementación y validación de un prototipo de simulación de eventos 
discretos. El artefacto computacional desarrollado modela la interacción de 
los parámetros logísticos clave ---capacidad de almacenamiento, políticas de 
inventario, demanda estocástica y, crucialmente, la frecuencia y duración de 
las interrupciones de la ruta de suministro--- concentrados en el nodo de 
almacenamiento primario de Coyhaique, que actúa como centro neurálgico del 
sistema regional.

El objetivo es crear un laboratorio virtual que permita cuantificar la 
resiliencia del sistema bajo diferentes escenarios. Mediante un diseño de 
experimentos formal, se evaluará la sensibilidad del sistema a distintos 
parámetros, buscando confirmar la hipótesis de que la resiliencia es 
significativamente más sensible a la duración de las disrupciones de ruta 
que a las variaciones en la capacidad de almacenamiento. El prototipo 
validado sienta una base metodológica para la toma de decisiones informadas, 
instrumentalizando una recomendación explícita de la política pública 
regional y contribuyendo al fortalecimiento de la seguridad energética de 
Aysén.

\vspace{1cm}
\noindent\textbf{Palabras Clave:} Simulación de Eventos Discretos, 
Resiliencia de Cadenas de Suministro, Gestión de Inventarios, Análisis de 
Riesgos, Seguridad Energética, Ingeniería de Sistemas.