\chapter{Introducción}
\label{chap:introduccion}

La Región de Aysén depende del 100\% de Gas Licuado de Petróleo (GLP) importado vía terrestre desde Argentina. El sistema opera con tres distribuidores (Abastible, Lipigas, Gasco) que almacenan 431 toneladas en Coyhaique, el nodo logístico central. Esta capacidad proporciona 8,2 días de autonomía para la demanda regional.

El suministro enfrenta disrupciones recurrentes de la Ruta 7 (única vía de acceso): cierres por nevadas, derrumbes y conflictos sociales en Argentina. El informe CIEP 2025 documenta una frecuencia de 4 eventos/año, con disrupciones históricas de hasta 21 días (conflicto Argentina 2021). Esta realidad genera un gap estructural: 8,2 días de autonomía vs. disrupciones de hasta 21 días.

Este trabajo desarrolla un modelo de simulación de eventos discretos para cuantificar la resiliencia del sistema bajo diferentes escenarios de capacidad de almacenamiento y duración de disrupciones. El modelo permite comparar la Propuesta 10.4 de Gasco (expansión de capacidad a 681 TM por \$1,5M USD) vs. estrategias de mitigación de disrupciones (rutas alternativas, protocolos con Argentina).

\textbf{Estructura del documento:} Capítulo \ref{chap:planteamiento-problema} caracteriza el sistema y sus vulnerabilidades. Capítulo \ref{chap:hipotesis} plantea la pregunta de investigación: ¿cuál factor (capacidad vs. disrupciones) domina la resiliencia del sistema? Capítulos \ref{chap:marco-teorico} y \ref{chap:estado-del-arte} presentan fundamentos teóricos. Capítulo \ref{chap:metodologia} detalla el diseño del experimento Monte Carlo (60.000 simulaciones). Capítulos \ref{chap:resultados} y \ref{chap:discusion} presentan resultados y análisis.