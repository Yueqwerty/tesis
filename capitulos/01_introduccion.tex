\chapter{Introducción}
\label{chap:introduccion}

\section{Contexto General}

La gestión de cadenas de suministro en zonas geográficamente aisladas representa un desafío de ingeniería que excede las soluciones logísticas convencionales. A diferencia de los sistemas conectados a redes troncales robustas, donde la variabilidad se amortigua mediante redundancias múltiples, las zonas aisladas operan bajo condiciones de fragilidad estructural. En estos entornos, la continuidad del servicio no depende únicamente de la capacidad de almacenamiento estático, sino de la dinámica temporal de reabastecimiento frente a interrupciones estocásticas.

La literatura en ingeniería de sistemas define la resiliencia no como la ausencia de fallas, sino como la capacidad de un sistema para absorber perturbaciones y recuperar su estado operativo. Sin embargo, en el contexto energético, las herramientas tradicionales de planificación suelen basarse en modelos deterministas —promedios anuales y márgenes de seguridad estáticos— que fallan al capturar la complejidad de eventos de "cola pesada", como disrupciones climáticas extremas o cortes de ruta prolongados.

\section{Contexto Regional}

La Región de Aysén presenta un caso paradigmático de aislamiento energético extremo. Sin conexión terrestre continua con el resto de Chile y dependiente de rutas marítimas y carreteras vulnerables a factores climáticos, la región importa el 100% de sus combustibles fósiles. El Gas Licuado de Petróleo (GLP) no es aquí un combustible suntuario, sino un insumo crítico de subsistencia, utilizado masivamente para calefacción residencial en un clima donde las temperaturas invernales descienden frecuentemente bajo cero.

El sistema opera bajo un oligopolio de distribución conformado por tres actores principales: \textbf{Abastible, Lipigas y Gasco}. Estas empresas concentran su inventario en un único nodo logístico central en Coyhaique, con una capacidad de almacenamiento combinada de \textbf{431 toneladas métricas (TM)}. Esta infraestructura debe abastecer una demanda regional que exhibe una fuerte estacionalidad, alcanzando picos en invierno que ponen a prueba la capacidad del sistema.

La cadena logística implica un transporte multimodal complejo: desembarque marítimo en puertos intermedios (Puerto Chacabuco o Puerto Cisnes) y transporte terrestre vía Ruta 7. Esta configuración crea un sistema con un "punto único de falla" (Single Point of Failure): si el inventario en el nodo central se agota debido a un bloqueo en la ruta de suministro, no existen sustitutos inmediatos viables, desencadenando una crisis de seguridad energética regional.

\section{Problemática}

El problema central no es la falta de combustible en el mercado global, sino la incapacidad del sistema logístico regional para garantizar la disponibilidad continua frente a la variabilidad conjunta de la demanda y el suministro.

Los datos técnicos, corroborados por el informe \cite{CNE2024}, revelan una brecha estructural alarmante: la capacidad instalada de 431 TM proporciona una autonomía teórica de apenas \textbf{8,2 días} bajo condiciones de demanda invernal. Sin embargo, la historia de disrupciones de la Ruta 7 muestra eventos de bloqueo —causados por nevadas, derrumbes o conflictos gremiales— que pueden extenderse hasta por \textbf{21 días}, como se evidenció durante las protestas en Argentina en 2021.

Esta disparidad (8,2 días de autonomía vs. 21 días de riesgo de corte) define la vulnerabilidad del sistema. Actualmente, la planificación se realiza mediante herramientas estáticas que ignoran esta estocasticidad, resultando en un sistema reactivo donde las medidas de mitigación se activan solo cuando la crisis es inminente.

\section{Brecha Detectada}

Existe una desconexión fundamental entre la complejidad del problema físico (un sistema dinámico, estocástico y no lineal) y las herramientas utilizadas para resolverlo (modelos estáticos lineales).

Mientras que la ingeniería industrial moderna utiliza ampliamente la Simulación de Eventos Discretos (DES) para optimizar líneas de producción y logística portuaria, su aplicación en la planificación de seguridad energética regional en Chile es incipiente. No existe actualmente un modelo computacional validado que permita a los tomadores de decisiones en Aysén responder preguntas del tipo: "¿Cuál es la probabilidad de desabastecimiento si la demanda crece un 10% y ocurre un bloqueo de ruta de 7 días en julio?". La falta de esta herramienta obliga a tomar decisiones de inversión basándose en intuición, aumentando el riesgo de falla de servicio.

\section{Requerimientos de la Solución}

Para cerrar esta brecha, se requiere una solución desde la ingeniería informática que permita modelar la complejidad temporal del sistema. No basta con una ecuación cerrada; se necesita un artefacto de software capaz de simular el comportamiento del sistema a través del tiempo.

La solución debe satisfacer los siguientes requerimientos técnicos para ser efectiva:
1.  **Capacidad de Simulación Estocástica:** El software debe implementar algoritmos de generación de números pseudoaleatorios para modelar la incertidumbre en la demanda (ruido gaussiano) y en las disrupciones de ruta (procesos de llegada de Poisson).
2.  **Granularidad Temporal:** El modelo debe operar con un reloj de simulación discreto capaz de evaluar el estado del sistema día a día, capturando la acumulación y agotamiento de inventarios de forma dinámica.
3.  **Parametrización Flexible:** Debe permitir configurar distintos escenarios (aumento de capacidad de tanques a 681 TM, cambios en la flota, variación de políticas de stock) sin reescribir el código fuente.
4.  **Reproducibilidad:** A diferencia de los análisis ad-hoc, la solución debe ser un software ejecutable que permita correr experimentos de Monte Carlo (miles de iteraciones) para obtener resultados estadísticamente significativos.

En resumen, se propone el desarrollo de un simulador de eventos discretos ("Digital Twin" simplificado) que transforme la incertidumbre inherente del entorno en métricas de riesgo cuantificables.
