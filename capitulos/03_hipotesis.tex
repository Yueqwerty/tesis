\chapter{Pregunta de Investigación e Hipótesis}
\label{chap:hipotesis}

El Capítulo \ref{chap:planteamiento-problema} identificó que el sistema de suministro de GLP en Aysén opera con dos tipos de vulnerabilidades:

\begin{itemize}
    \item \textbf{Vulnerabilidad exógena:} Disrupciones de la Ruta 7 (frecuencia: 4 eventos/año, duración: hasta 21 días).
    \item \textbf{Vulnerabilidad endógena:} Capacidad de almacenamiento limitada (431 TM = 8,2 días de autonomía).
\end{itemize}

Este capítulo plantea la pregunta de investigación y la hipótesis central que guían el diseño del modelo de simulación y el experimento Monte Carlo.

\section{Pregunta de Investigación}
\label{sec:pregunta-investigacion}

¿Cuál de los dos factores (duración de disrupciones vs. capacidad de almacenamiento) tiene mayor impacto en la resiliencia del sistema?

Específicamente: ¿Cómo diseñar un modelo de simulación para cuantificar la sensibilidad relativa del nivel de servicio a estos dos factores?

\section{Hipótesis Central}
\label{sec:hipotesis}

\textbf{Hipótesis:} La resiliencia del sistema de suministro de GLP de Aysén es significativamente más sensible a la duración de las disrupciones (factor exógeno) que a la capacidad de almacenamiento (factor endógeno).

\subsection{Justificación de la hipótesis}

Esta hipótesis se basa en dos observaciones del sistema actual:

\begin{enumerate}
    \item \textbf{Disrupciones largas superan la autonomía del sistema:} La capacidad actual (431 TM) proporciona 8,2 días de autonomía. Disrupciones documentadas de hasta 21 días (conflicto Argentina 2021) exceden esta capacidad por un factor de 2,5×.

    \item \textbf{Expansión de capacidad tiene retornos decrecientes:} La Propuesta 10.4 de Gasco incrementa capacidad en 58\% (de 431 a 681 TM), pero la autonomía solo aumenta a 13 días. Disrupciones de 21 días seguirían generando quiebres de stock.
\end{enumerate}

\subsection{Operacionalización de la hipótesis}

La hipótesis se prueba mediante un experimento factorial $2 \times 3$ que compara:

\textbf{Factor endógeno (capacidad):}
\begin{itemize}
    \item Status Quo: 431 TM
    \item Propuesta: 681 TM (+58\%)
\end{itemize}

\textbf{Factor exógeno (duración de disrupciones):}
\begin{itemize}
    \item Corta: Triangular(3, 3.5, 7) días
    \item Media: Triangular(3, 7, 14) días
    \item Larga: Triangular(3, 10.5, 21) días
\end{itemize}

La hipótesis se confirma si el ratio de sensibilidad $\frac{\text{Efecto exógeno}}{\text{Efecto endógeno}} > 1$.

\subsection{Implicaciones para política pública}

Si la hipótesis se confirma, las inversiones en mitigación de disrupciones (rutas alternativas, protocolos con Argentina, mejora de Ruta 7) tendrían mayor retorno en resiliencia que la inversión de \$1,5M USD en expansión de capacidad (Propuesta 10.4).