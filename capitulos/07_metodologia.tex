\chapter{Diseño Metodológico}
\label{chap:metodologia}

Este capítulo detalla el plan de trabajo y los métodos que se emplearán para
alcanzar los objetivos de la investigación. El enfoque se enmarca en las
disciplinas de la Ciencia Computacional y la Ingeniería de Sistemas, adoptando
un proceso sistemático para el diseño, implementación y evaluación del modelo
de simulación. El propósito es establecer un marco de trabajo riguroso,
transparente y replicable.

\section{Fase 1: Modelado Conceptual del Sistema (Objetivo 1)}
\label{sec:conceptual-modeling}

El objetivo de esta fase es la abstracción formal del sistema real en un 
modelo conceptual que sirva como especificación para la implementación.

% --- SECCIÓN REESCRITA PARA MÁXIMA CLARIDAD ---
\subsection{Definición de Límites y Supuestos del Modelo}
\label{sec:limites-supuestos}

Para abordar el problema regional de manera cuantitativa, es imperativo 
establecer un límite de sistema (\textit{system boundary}) claro para el 
modelo computacional. Si bien el impacto de una disrupción en el suministro 
de GLP es de alcance regional, el centro de gravedad logístico y el punto 
de control más crítico de la cadena se encuentra en el nodo de 
almacenamiento primario de Coyhaique.

Por consiguiente, el modelo se centrará en la dinámica de inventarios de 
las tres plantas mayoristas ubicadas en dicho nodo. Se modelará el flujo de 
suministro desde los puntos de origen (Neuquén y Cabo Negro) hasta estas 
plantas, y la salida de producto se modelará como una demanda agregada que 
representa el consumo de la zona de influencia directa.

Este enfoque se justifica porque la resiliencia del nodo de Coyhaique actúa 
como un \textbf{proxy directo} de la resiliencia de toda la región. Un 
quiebre de stock en este punto neurálgico implica, por definición, la 
incapacidad de abastecer a las demás localidades. La modelización de la red 
de distribución de última milla y las dinámicas de inventario en otros puntos 
de la región quedan fuera del alcance de este estudio y se plantean como 
líneas de trabajo futuro.

\section{Enfoque de la Investigación}
\label{sec:research-approach}

Esta investigación adopta un enfoque cuantitativo basado en simulación como
método de experimentación computacional. Dada la naturaleza estocástica y
compleja de la cadena de suministro de GLP, los modelos puramente analíticos
son insuficientes. La simulación permite probar hipótesis y evaluar políticas
bajo condiciones controladas, algo inviable en el sistema real. El desarrollo
se estructura en las tres fases ilustradas en la
Figura~\ref{fig:conceptual-diagram-detailed}.

\begin{figure}[htbp]
    \centering
    \begin{tikzpicture}[
        font=\sffamily\small,
        scale=0.95,
        node distance=5cm,
        box/.style = {rectangle, draw=black!50, line width=1.5pt, rounded corners=3pt,
                     minimum height=2cm, minimum width=3cm,
                     text centered, font=\sffamily\small, align=center},
        arrow/.style = {->, >=Stealth, line width=2pt},
        dashed arrow/.style = {->, >=Stealth, dashed, line width=1pt, opacity=0.7},
        label/.style = {font=\sffamily\footnotesize, fill=white, inner sep=1pt}
    ]

        % Componentes principales - clean
        \node[box, fill={rgb,255:red,127;green,161;blue,169}, fill opacity=0.2] (A) at (0,0) {
            Política\\(Q,R)
        };

        \node[box, fill={rgb,255:red,143;green,193;blue,169}, fill opacity=0.2] (B) at (4.5,0) {
            Transporte\\(LT = 6d)
        };

        \node[box, fill={rgb,255:red,180;green,167;blue,214}, fill opacity=0.2] (C) at (9,0) {
            Almacenamiento\\(431/681 TM)
        };

        \node[box, fill={rgb,255:red,244;green,213;blue,141}, fill opacity=0.2] (D) at (13.5,0) {
            Demanda\\(52.5 TM/d)
        };

        % Flujo principal - clean
        \draw[arrow, color={rgb,255:red,127;green,161;blue,169}] (A) -- (B);
        \draw[arrow, color={rgb,255:red,127;green,161;blue,169}] (B) -- (C);
        \draw[arrow, color={rgb,255:red,127;green,161;blue,169}] (C) -- (D);

        % Retroalimentación
        \draw[arrow, color={rgb,255:red,111;green,179;blue,184}, dashed]
            (C.south) .. controls (4.5,-1.2) and (2,-1.2) .. (A.south);

        % Disrupciones - clean
        \node[box, fill={rgb,255:red,247;green,161;blue,183}, fill opacity=0.15,
              minimum height=2cm] (E) at (2.2,-3) {
            Disrupciones\\$\lambda=4$/año
        };

        \draw[arrow, color={rgb,255:red,200;green,106;blue,83}]
            (E.north) -- (A.south);
        \draw[arrow, color={rgb,255:red,200;green,106;blue,83}]
            (E.north east) -- (B.south);

        % Parámetros
        \node[box, fill=gray!10,
              minimum height=2cm] (F) at (9,-3) {
            Parámetros\\Configurables
        };

        \draw[dashed arrow, gray!60] (F.north) -- (C.south);

    \end{tikzpicture}
    \caption{Modelo Conceptual del Sistema de Distribución de GLP en Aysén. El diagrama muestra el flujo operacional (azul), las disrupciones exógenas (rosa/terracota), los parámetros configurables (gris) y la métrica de resiliencia (borgoña). La política $(Q,R)$ controla el reabastecimiento basándose en el nivel de inventario.}
    \label{fig:conceptual-diagram-detailed}
\end{figure}

\subsection{Parametrización y Modelado Estocástico}

Los parámetros del modelo se dividen en dos categorías: deterministas,
tomados directamente del informe \cite{CIEP2025}, y estocásticos, modelados
mediante distribuciones de probabilidad calibradas con datos históricos.

\subsubsection{Parámetros Deterministas}

\begin{itemize}
    \item \textbf{Capacidad de Almacenamiento:}
    \begin{itemize}
        \item Status Quo: \SI{431}{TM} (Abastible: 150, Lipigas: 240, Gasco: 41)
        \item Propuesta 10.4: \SI{681}{TM} (incremento de \SI{250}{TM})
    \end{itemize}

    \item \textbf{Política de Inventario $(Q, R)$:}
    \begin{itemize}
        \item Punto de Reorden ($R$): 50\% de la capacidad
        \item Cantidad de Pedido ($Q$): 50\% de la capacidad
        \item Inventario Inicial: 60\% de la capacidad (arranque realista)
    \end{itemize}

    \item \textbf{Lead Time Nominal:} \SI{6}{días} (tiempo promedio de entrega
    desde Cabo Negro o Neuquén hasta Coyhaique)

    \item \textbf{Horizonte de Simulación:} \SI{365}{días} (1 año)
\end{itemize}

\subsubsection{Variables Estocásticas y Distribuciones}

Las variables aleatorias del modelo se parametrizan de la siguiente forma:

\begin{enumerate}
    \item \textbf{Frecuencia de Disrupciones:} Se modela como un proceso de
    Poisson con tasa $\lambda = 4$ eventos/año. El tiempo entre disrupciones
    consecutivas sigue una distribución \textbf{Exponencial}:
    \[ T_{\text{entre}} \sim \text{Exp}\left(\frac{\lambda}{365}\right) = \text{Exp}(0.0110) \]
    donde $\lambda$ corresponde a la frecuencia de Nivel 4 identificada en la
    matriz de riesgos~\cite{CIEP2025}.

    \item \textbf{Duración de Disrupciones:} Se emplea una distribución
    \textbf{Triangular}($a$, $b$, $c$) calibrada con datos históricos:
    \[ D_{\text{disrup}} \sim \text{Triangular}(a, b, c) \]
    Los parámetros varían según el escenario experimental:
    \begin{itemize}
        \item Corta: $a = 3$, $b = 3.5$, $c = 7$ días
        \item Media: $a = 3$, $b = 7$, $c = 14$ días
        \item Larga: $a = 3$, $b = 10.5$, $c = 21$ días (conflicto Argentina 2021)
    \end{itemize}
    La distribución triangular permite modelar la incertidumbre con parámetros
    interpretables: mínimo histórico, valor más probable, y máximo observado.

    \item \textbf{Demanda Diaria:} Se modela como un proceso estocástico con
    componente estacional y ruido:
    \[ D(t) = D_{\text{base}} \cdot \left(1 + 0.25 \sin\left(\frac{2\pi(t - 172)}{365}\right)\right) \cdot \epsilon(t) \]
    donde:
    \begin{itemize}
        \item $D_{\text{base}} = \SI{52.5}{TM/día}$ (demanda promedio del mes
        de mayor consumo)
        \item El término sinusoidal modela la estacionalidad invernal
        (pico en julio, día $\approx 200$)
        \item $\epsilon(t) \sim \mathcal{N}(1.0, 0.15)$ es el ruido estocástico
        diario ($\pm 15\%$ de variabilidad)
    \end{itemize}
    La demanda base de \SI{52.5}{TM/día} se calibró para representar el escenario
    de estrés del sistema (mes de mayor consumo), lo que genera una autonomía
    conservadora de $\approx \SI{5}{días}$ en el escenario Status Quo, frente a
    los \SI{8.2}{días} calculados con demanda promedio anual.
\end{enumerate}

\subsubsection{Generación de Números Aleatorios}

Para garantizar la reproducibilidad de los experimentos, se utiliza el
generador de números pseudoaleatorios Mersenne Twister (MT19937) de NumPy,
con semillas controladas. Cada réplica $r$ de la configuración $c$ emplea
una semilla única:
\[ s_{c,r} = s_{\text{base}} + (c - 1) \times 1000 + r \]
donde $s_{\text{base}} = 42$. Esta estrategia asegura independencia
estadística entre réplicas y reproducibilidad exacta de los resultados.

\section{Fase 2: Implementación del Prototipo (Objetivo 2)}
\label{sec:implementation}

\subsection{Stack Tecnológico y Arquitectura de Software}
La selección de herramientas busca maximizar la flexibilidad y la 
reproducibilidad científica:

\begin{itemize}
    \item \textbf{Núcleo:} \textbf{Python 3.x} y \textbf{SimPy}, por su 
    capacidad para modelar procesos complejos de forma nativa.
    
    \item \textbf{Análisis y Visualización:} \textbf{Pandas}, \textbf{Matplotlib}, 
    y \textbf{Seaborn} para el procesamiento de datos y la generación de 
    gráficos.
    
    \item \textbf{Reproducibilidad:} Se utilizará \textbf{Git} para el control 
    de versiones y un gestor de entornos virtuales (\textbf{venv} o 
    \textbf{Conda}) para encapsular las dependencias.
\end{itemize}

La arquitectura del software seguirá un patrón \textbf{Modelo-Experimento-Análisis}, 
separando la lógica de la simulación (\texttt{modelo.py}), de la orquestación de 
escenarios (\texttt{experimento.py}) y de los scripts de análisis (\texttt{analisis.py}). 
Los parámetros se externalizarán a un archivo \texttt{config.json}.

\section{Fase 3: Evaluación y Experimentación (Objetivo 3)}
\label{sec:validation-experimentation}

\subsection{Protocolo de Verificación y Validación (V\&V)}
Se aplicará un protocolo formal para establecer la credibilidad del modelo:

\begin{itemize}
    \item \textbf{Verificación:} Se realizarán \textit{code walkthroughs} y 
    pruebas de componentes deterministas para asegurar que el código refleja 
    el modelo conceptual.
    
    \item \textbf{Validación:} Se empleará un enfoque de múltiples facetas:
    \begin{itemize}
        \item \textbf{Validación de Datos Históricos:} Se compararán las 
        distribuciones estadísticas (media, varianza) de las métricas 
        clave del modelo (ej. días de autonomía) con los valores de 
        referencia del informe \cite{CIEP2025} (ej. media de 8.2 días).
        
        \item \textbf{Validación por Juicio de Expertos (Face Validity):} 
        Se realizarán ``pruebas de Turing'' para modelos, donde se 
        presentarán trazas de salida del modelo a los expertos técnicos 
        de la SEC para que evalúen su plausibilidad operativa.
    \end{itemize}
\end{itemize}

\subsection{Diseño de Experimentos (DoE)}

Para probar la hipótesis central, se ejecutará un \textbf{Experimento Monte Carlo}
con diseño factorial $2 \times 3$. Se realizarán 1,000 réplicas independientes
para cada combinación de factores, totalizando 6,000 simulaciones. Este tamaño
de muestra garantiza:

\begin{itemize}
    \item Estimación precisa de las medias poblacionales (error estándar < 0,2\%),
    \item Intervalos de confianza al 95\% con amplitud reducida,
    \item Potencia estadística > 0,95 para detectar diferencias de 1\% entre medias,
    \item Validación robusta de supuestos de normalidad mediante tests formales.
\end{itemize}

\begin{table}[htbp]
    \centering
    \caption{Diseño Experimental Monte Carlo para Evaluación de Resiliencia.}
    \label{tab:doe}
    \begin{tabular}{@{}ll@{}}
        \toprule
        \textbf{Componente} & \textbf{Especificación} \\ \midrule
        \textbf{Método} & Experimento Monte Carlo \\
        \addlinespace
        \textbf{Factores} & 1. Capacidad de Almacenamiento (Endógeno) \\
                          & 2. Duración de Disrupción (Exógeno) \\
        \addlinespace
        \textbf{Niveles Factor 1} & Nivel 1: Status Quo (431 TM) \\
                                 & Nivel 2: Propuesta 10.4 (681 TM) \\
        \addlinespace
        \textbf{Niveles Factor 2} & Nivel 1: Corta (7 días máximo) \\
                                 & Nivel 2: Media (14 días máximo) \\
                                 & Nivel 3: Larga (21 días máximo) \\
        \addlinespace
        \textbf{Réplicas} & 1,000 por configuración \\
        \textbf{Total simulaciones} & 6,000 \\
        \addlinespace
        \textbf{Variables de Respuesta} & 1. Nivel de Servicio (\%) \\
        (KPIs)                          & 2. Probabilidad de Quiebre de Stock \\
                                       & 3. Días con Quiebre \\
                                       & 4. Inventario Promedio \\ \bottomrule
    \end{tabular}
\end{table}

\subsection{Análisis Estadístico}

Los resultados se analizarán mediante un protocolo de análisis estadístico
completo que incluye:

\begin{enumerate}
    \item \textbf{Estadística Descriptiva:} Media, mediana, desviación estándar,
    y cuartiles para cada configuración. Intervalos de confianza al 95\%
    calculados mediante método bootstrap.

    \item \textbf{Análisis de Varianza (ANOVA):} ANOVA de dos vías para
    determinar la significancia estadística de los efectos principales de
    cada factor y de sus interacciones.

    \item \textbf{Validación de Supuestos:} Tests de normalidad (Shapiro-Wilk)
    y Q-Q plots para validar supuestos paramétricos. Tests de homogeneidad
    de varianzas (Levene).

    \item \textbf{Análisis de Sensibilidad:} Cuantificación de la sensibilidad
    del nivel de servicio a cada factor, expresada como cambio absoluto en
    puntos porcentuales.
\end{enumerate}

Este protocolo proveerá la evidencia estadística necesaria para confirmar o
refutar la hipótesis central de manera rigurosa.