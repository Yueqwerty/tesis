\chapter{Estado del Arte}
\label{chap:estado-del-arte}

Para establecer la originalidad y pertinencia de esta investigación, es 
necesario situarla en el contexto del conocimiento y la práctica existentes. 
Este capítulo realiza una revisión crítica en dos frentes: primero, el estado 
de la práctica en el uso de la simulación para la gestión de riesgos 
logísticos a nivel global; y segundo, el estado del diagnóstico técnico 
sobre la vulnerabilidad del suministro energético en Aysén. El objetivo es 
demostrar que, si bien ambas áreas están desarrolladas por separado, existe 
una brecha significativa en su intersección, la cual este proyecto se 
propone cerrar.

\section{La Simulación como Estándar en la Gestión de Riesgos Logísticos}
\label{sec:soa-simulation-apps}

La simulación de eventos discretos (SED) no es una metodología académica 
incipiente, sino una herramienta consolidada y considerada el estándar de 
oro para el análisis y la optimización de cadenas de suministro complejas. 
Su aplicación es transversal, desde el sector comercial hasta el humanitario.

En la industria, corporaciones con operaciones logísticas de alta intensidad 
como Amazon, FedEx o Intel, emplean modelos de simulación de forma rutinaria 
para la toma de decisiones estratégicas. Estos modelos se utilizan para el 
diseño de redes de distribución, la optimización de operaciones en centros 
de almacenamiento y, de manera crucial, para el análisis de riesgos y la 
planificación de la continuidad del negocio (\textit{Business Continuity 
Planning}). La simulación permite a estas organizaciones cuantificar el 
impacto de escenarios de disrupción ---fallas en infraestructura, huelgas, 
eventos climáticos extremos--- y evaluar la efectividad de planes de 
contingencia antes de que ocurran, optimizando así el trade-off entre costo 
y resiliencia \cite{Banks2010}.

De manera análoga, en el ámbito de la logística humanitaria, organizaciones 
como el Programa Mundial de Alimentos de las Naciones Unidas (WFP) utilizan 
la simulación para planificar la distribución de ayuda ante desastres 
naturales. Estos modelos son fundamentales para decidir sobre el 
pre-posicionamiento de inventarios críticos (alimentos, medicinas) y para 
optimizar las rutas de distribución en entornos caóticos y con 
infraestructura dañada \cite{Law2015}. Estos casos de estudio demuestran que 
la simulación es la metodología por excelencia para analizar sistemas donde 
la incertidumbre, la complejidad y el riesgo son factores dominantes.

\section{El Diagnóstico de Vulnerabilidad en Aysén: Un Análisis Estático}
\label{sec:soa-aysen-report}

El informe ``Investigación Vulnerabilidad de Suministro de GLP y 
Combustibles Líquidos'' \cite{CIEP2025} constituye el diagnóstico técnico 
más completo y actualizado del sistema bajo estudio. Este documento provee 
una ``anatomía'' detallada de la cadena de suministro, pero no un análisis 
de su ``fisiología'' bajo estrés. Sus contribuciones y limitaciones pueden 
resumirse en:
\begin{itemize}
    \item \textbf{Fortaleza (Caracterización Exhaustiva):} El informe realiza 
    un análisis descriptivo profundo de la demanda, las rutas, las 
    capacidades de almacenamiento y los actores del sistema. Provee los 
    datos agregados y las descripciones cualitativas que sirven como base 
    indispensable para la parametrización de cualquier modelo cuantitativo.
    \item \textbf{Limitación (Naturaleza Estática):} Su principal herramienta 
    analítica es una matriz de riesgos que evalúa probabilidad e impacto de 
    forma aislada. Este enfoque, si bien es útil para la identificación y 
    priorización inicial, es inherentemente estático. No puede capturar la 
    dinámica temporal del sistema, los efectos en cascada de una disrupción, 
    ni la interacción no lineal entre los parámetros (ej. cómo una 
    disrupción de 15 días impacta de forma diferente si ocurre en temporada 
    de alta o baja demanda).
\end{itemize}
En esencia, el informe responde a la pregunta de qué puede fallar, pero 
no provee una herramienta para responder a las preguntas de cuánto, 
cuándo y con qué probabilidad el sistema completo fallará.

\section{Identificación del Vacío de Conocimiento y Contribución Original}
\label{sec:soa-gap}

La revisión del estado del arte revela una situación paradójica: por un 
lado, la práctica industrial y académica ha establecido la simulación como 
la herramienta idónea para analizar la resiliencia de cadenas de suministro; 
por otro, el diagnóstico más completo sobre la cadena de suministro de GLP 
en Aysén carece de este tipo de análisis dinámico.

El vacío de conocimiento que este proyecto aborda, por lo tanto, no es de 
naturaleza teórica, sino de aplicación metodológica a un problema 
específico, validado y de alta criticidad. La contribución original de 
esta tesis radica en desarrollar el artefacto computacional que cierra 
esa brecha.

De manera explícita, este trabajo ejecuta la iniciativa de gestión 11.11: 
``Simulación de emergencias energéticas'', propuesta en el propio informe 
de referencia \cite{CIEP2025}. El informe califica la simulación como una 
``buena práctica'' para la región y establece la meta de ``Realizar cada 2-3 
años un ejercicio de simulación''. Mientras el informe propone esta 
necesidad, y la autoridad regional la valida al identificar la urgencia de 
un ``mecanismo que permita hacer un monitoreo permanente y en tiempo real'' 
(Laibe, T., comunicación personal, 11 de junio, 2025), este proyecto de 
tesis diseña, implementa y valida el artefacto que permite llevar a cabo 
dicha tarea.

Adicionalmente, el proyecto aborda una brecha de capacidad institucional 
identificada por la propia autoridad regional. La falta de personal técnico 
especializado para el monitoreo continuo de los hidrocarburos amplifica la 
necesidad de una herramienta como la propuesta, que no solo permite análisis 
más sofisticados, sino que también actúa como un multiplicador de 
capacidades, encapsulando conocimiento experto para potenciar la toma de 
decisiones de los equipos de gestión existentes.