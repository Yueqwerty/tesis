\chapter{Planteamiento del Problema}
\label{chap:planteamiento-problema}

El suministro de recursos energéticos en regiones aisladas constituye un
problema complejo de alta relevancia. La Región de Aysén, en la Patagonia
chilena, ilustra esta problemática en su sistema de suministro de Gas Licuado
de Petróleo (GLP). Este capítulo formaliza los elementos que configuran un
escenario de riesgo sistémico. Se argumenta que la interacción de tres
factores ---alta dependencia energética, topología logística frágil y vacíos
en la planificación estratégica--- demanda un nuevo enfoque analítico que los
métodos estáticos actuales no pueden proveer.

\section{Caracterización del Sistema}
\label{sec:caracterizacion-sistema}

El sistema de suministro de GLP en Aysén opera bajo condiciones que lo
definen como un sistema complejo y crítico. Su criticidad se deriva de la
confluencia de una demanda energética elevada, una estructura de mercado
oligopólica y una infraestructura precaria.

\subsection{La Dimensión de la Demanda: Criticidad y Crecimiento Sostenido}

La demanda energética de la región es un factor distintivo a nivel nacional. 
Con un consumo per cápita de \SI{27.25}{Gcal}, que excede en un 65\% la 
media de Chile, la población depende intensivamente de un flujo energético 
constante para sostener funciones básicas. El GLP satisface nichos de 
consumo ---cocción de alimentos y agua caliente sanitaria--- para los cuales 
no existen sustitutos inmediatos a gran escala. Esta inelasticidad 
fundamental de la demanda convierte al GLP en un recurso de alta prioridad 
social, cuya interrupción tiene consecuencias directas e inmediatas sobre 
el bienestar de la población.

Adicionalmente, la demanda no es estática. El sistema enfrenta un 
crecimiento sostenido del \SI{3.8}{\percent} anual, impulsado por el 
desarrollo demográfico y económico de la región. A esto se suma un factor 
disruptivo futuro: la proyectada incorporación de una central térmica que 
consumirá \SI{14.4}{ton/día} de GLP. Este crecimiento y cambio estructural 
implican que la presión sobre la ya frágil cadena de suministro se 
intensificará de manera no lineal en los próximos años.

\subsection{La Dimensión del Suministro: Topología Lineal y Soberanía Comprometida}

% --- SECCIÓN PROFUNDIZADA CON ANÁLISIS DE LA RUTA ---
La estructura de la cadena de suministro, detallada en el informe de 
referencia~\cite{CIEP2025}\footnote{El informe, encargado por la Seremía de 
Energía de Aysén, fue finalizado y presentado en 2025. A la fecha de esta 
publicación, se encuentra en proceso de difusión pública.}, se caracteriza
por una topología logística 
lineal con una dependencia absoluta de fuentes exógenas. El informe es 
categórico al afirmar que ``La totalidad del gas licuado que llega a la 
región, lo hace vía camiones que transitan por el paso Huemules desde 
Argentina''. Este diseño, carente de redundancia, convierte a la ruta 
terrestre en un punto único de falla (\textit{Single Point of Failure}) de 
manual. La magnitud de esta dependencia se cuantifica en el hecho de que el 
86\% del recorrido terrestre desde la planta de Cabo Negro hasta Coyhaique 
transcurre por territorio argentino, lo que implica que la soberanía 
logística y la seguridad del suministro regional están estructuralmente 
comprometidas por la dinámica de un país vecino.

\section{Análisis de Vulnerabilidades: La Disonancia Crítica del Sistema}
\label{sec:analisis-vulnerabilidades}

Las vulnerabilidades del sistema no son meramente teóricas, sino que se 
manifiestan en una disonancia crítica: la magnitud de las amenazas exógenas 
supera con creces la capacidad de absorción de la fragilidad endógena del 
sistema.

\subsection{Vulnerabilidades Exógenas: Amenazas Recurrentes y de Larga Duración}

% --- SECCIÓN PROFUNDIZADA CON CUANTIFICACIÓN DEL RIESGO ---
El principal vector de riesgo es la incertidumbre estocástica en el 
aprovisionamiento. El informe de riesgos~\cite{CIEP2025} identifica dos 
amenazas dominantes. Primero, el evento de ``Nevadas / Cierre cruce 
fronterizo'' (Tag \#57), clasificado con una probabilidad de Nivel 4 (``Casi 
Seguro''), lo que implica una frecuencia de ocurrencia de ``al menos una vez 
cada tres meses''. Segundo, el riesgo de ``Conflicto social en Argentina'' 
(Tag \#45), que, aunque menos frecuente, ha demostrado la capacidad de 
paralizar el suministro por períodos de hasta tres semanas. Esta 
realidad es corroborada por la autoridad energética regional, quien 
identifica la ``conflictividad social'' y la ``mantención de los caminos'' 
en Argentina como la principal amenaza para la continuidad del suministro 
(Laibe, T., comunicación personal, 11 de junio, 2025). El sistema, por tanto, 
enfrenta amenazas recurrentes y de duración prolongada que introducen una 
alta variabilidad en el tiempo de entrega (\textit{lead time}).

\subsection{Vulnerabilidades Endógenas: Un Buffer Insuficiente y Estratégicamente Degradado}

% --- SECCIÓN PROFUNDIZADA CON ANÁLISIS DE LA DINÁMICA DE MERCADO ---
Frente a una amenaza documentada de 21 días, la capacidad de respuesta del
sistema es alarmantemente limitada. La infraestructura de almacenamiento
primario en Coyhaique posee una capacidad nominal total de solo
\textbf{\SI{431}{ton}}, distribuidas entre Abastible (\SI{150}{ton}), Lipigas (\SI{240}{ton}) y Gasco (\SI{41}{ton})~\cite{CIEP2025}. Esta capacidad se traduce en una autonomía teórica de apenas \textbf{\SI{8.2}{días}} para el mes de mayor venta.

Esta fragilidad no es solo una limitación física, sino el resultado de una 
dinámica de mercado adversa. El informe evidencia que ``por empresas los 
números son mucho más ajustados para Gasco que tiene una capacidad demasiado 
reducida para su nivel de ventas''~\cite{CIEP2025}. Este hecho es la 
manifestación práctica de un ``Dilema del Prisionero'': en un mercado 
oligopólico, la estrategia individualmente racional de minimizar costos de 
capital (manteniendo un inventario bajo) es adoptada por al menos un actor, 
lo que degrada la resiliencia colectiva y deja al sistema entero expuesto.

\section{Insuficiencia de los Marcos de Análisis y la Brecha Metodológica}
\label{sec:insuficiencia-marcos}

El marco de planificación regional y las herramientas de análisis existentes 
son inadecuados para gestionar la compleja interacción de las 
vulnerabilidades descritas.

Primero, existe un documentado vacío estratégico. La `Política Energética 
2050 Región de Aysén' centra sus esfuerzos de seguridad energética 
exclusivamente en el sector eléctrico, omitiendo un tratamiento formal para 
la cadena de suministro de combustibles.

Segundo, el diagnóstico técnico disponible~\cite{CIEP2025}, si bien es 
exhaustivo, es de naturaleza estática. Propone un portafolio de soluciones, 
como el aumento de la capacidad de almacenamiento en \SI{250}{ton} con una 
inversión estimada de \SI{1.5}{millones~de~USD}, pero no provee un método 
para evaluar su retorno en resiliencia. La matriz de riesgos evalúa 
probabilidad e impacto de forma aislada, pero no puede capturar los efectos 
en cascada de una disrupción a lo largo del tiempo.

% --- SECCIÓN PROFUNDIZADA CON EVIDENCIA OPERATIVA DE LA ENTREVISTA ---
Esta insuficiencia se materializa en la práctica operativa de la gestión de 
crisis regional. Como lo confirma la autoridad regional, dicha gestión se 
basa en protocolos reactivos y manuales, donde el proceso implica ``llamar a 
todas las empresas, construir un Excel y, más o menos, sacar una foto de 
cuánta disponibilidad de combustible hay'' (Laibe, T., comunicación personal, 
11 de junio, 2025). Este enfoque, calificado como ``rudimentario'', es la 
prueba fehaciente de la ausencia de una plataforma de análisis dinámico.

La brecha metodológica es, por tanto, clara: se carece de una herramienta 
que permita un análisis dinámico y cuantitativo para responder preguntas de 
ingeniería y gestión de inversiones, tales como:

\begin{quote}
    \textit{¿Cuál es la reducción porcentual en la probabilidad de quiebre 
    de stock que se obtiene de una inversión de \$\SI{1.5}{M}, bajo un 
    escenario de disrupciones con una frecuencia de Nivel 4 y una duración 
    estocástica, considerando además la demanda adicional de la nueva 
    central térmica?}
\end{quote}