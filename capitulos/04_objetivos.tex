% --- SIN CAMBIOS, YA ES CORRECTO ---
\chapter{Objetivos}
\label{chap:objetivos}

Para abordar la brecha metodológica identificada en el \cref{chap:planteamiento-problema} 
y responder a la pregunta de investigación formulada, el presente proyecto se 
estructura en torno a un conjunto de objetivos jerárquicos. Estos objetivos 
guían el proceso de diseño, desarrollo y validación del artefacto 
computacional propuesto, asegurando un enfoque sistemático para la creación 
de un laboratorio virtual que permita el análisis dinámico de la resiliencia 
del sistema de suministro de GLP.

\section{Objetivo General}
\label{sec:objetivo-general}

Diseñar un prototipo de simulación de eventos discretos, validado 
computacionalmente, para cuantificar el impacto de parámetros logísticos 
críticos sobre la resiliencia del sistema de suministro de GLP en el nodo 
Coyhaique.

\section{Objetivos Específicos}
\label{sec:objetivos-especificos}

La consecución del objetivo general se desglosa en tres fases metodológicas 
secuenciales, cada una representada por un objetivo específico que define 
una etapa clave del proceso de investigación y desarrollo:

\begin{enumerate}
    \item \textbf{Modelar} la dinámica conceptual de la cadena de suministro 
    de GLP en Coyhaique, realizando una abstracción formal de sus 
    componentes, procesos y parámetros clave a partir de la información 
    técnica disponible.
    
    \item \textbf{Implementar} un prototipo computacional del modelo de 
    simulación, desarrollando una arquitectura de software parametrizable 
    que traduzca el modelo conceptual a un artefacto funcional y 
    experimental.
    
    \item \textbf{Evaluar} la resiliencia del sistema mediante la ejecución 
    de un diseño experimental sobre el prototipo validado, utilizando la 
    simulación para generar la evidencia empírica necesaria para confirmar 
    o refutar la hipótesis de trabajo.
\end{enumerate}