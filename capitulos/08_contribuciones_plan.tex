\chapter{Conclusiones y Proyección del Trabajo}
\label{chap:conclusiones}

Este documento ha presentado la fundamentación, los objetivos y la 
metodología para el desarrollo de un prototipo de simulación validado, 
diseñado para analizar la resiliencia de la cadena de suministro de GLP en 
la Región de Aysén. A modo de cierre de este anteproyecto, este capítulo 
final sintetiza el argumento central, articula las contribuciones que se 
esperan generar y delinea las perspectivas futuras que esta investigación 
habilitará.

\section{Síntesis del Problema y la Solución Propuesta}

Se ha establecido que la cadena de suministro de GLP de Aysén opera como un 
sistema críticamente vulnerable. Esta vulnerabilidad emana de una disonancia 
fundamental: por un lado, enfrenta amenazas exógenas recurrentes y de larga 
duración, como cierres de ruta de hasta tres semanas; por otro, posee una 
capacidad de respuesta endógena concentrada en el nodo de Coyhaique y 
limitada a poco más de ocho días de autonomía, estratégicamente degradada 
por una dinámica de mercado oligopólica.

Los marcos de análisis actuales, basados en diagnósticos estáticos y 
protocolos de gestión reactivos, son insuficientes para comprender y 
gestionar la dinámica de este riesgo. Frente a esta brecha metodológica, 
este proyecto propone el diseño, implementación y validación de un 
artefacto computacional de simulación de eventos discretos. Este 
laboratorio virtual permitirá modelar la interacción de las variables del 
sistema y cuantificar su comportamiento bajo estrés, superando las 
limitaciones del análisis estático.

\section{Contribuciones Esperadas}

Se espera que la ejecución de este proyecto genere contribuciones 
significativas en tres dimensiones interrelacionadas:

\begin{description}
    \item[Contribución Metodológica:] Se introducirá un paradigma de 
    análisis dinámico y estocástico en un dominio actualmente evaluado con 
    herramientas estáticas. El prototipo permitirá pasar de la identificación 
    de riesgos a la cuantificación de la resiliencia, proveyendo un marco 
    para evaluar el sistema no en su estado promedio, sino en sus extremos.

    \item[Contribución Práctica Regional:] Se entregará una herramienta de 
    apoyo a la toma de decisiones para actores clave como la Seremía de 
    Energía y la SEC. El prototipo permitirá evaluar el ``retorno en 
    resiliencia'' de inversiones en infraestructura (ej. aumento de 
    almacenamiento), proveyendo una base empírica para la asignación de 
    recursos y la formulación de políticas públicas.

    \item[Contribución Institucional:] Al encapsular conocimiento experto y 
    una metodología de análisis compleja en un artefacto de software usable, 
    el proyecto actúa como un multiplicador de capacidades. Aborda la brecha 
    de capital humano técnico identificada por la autoridad regional, 
    potenciando la capacidad de análisis de los equipos de gestión existentes.
\end{description}

\section{Limitaciones y Líneas de Trabajo Futuro}

Todo modelo es, por definición, una simplificación de la realidad. Como se 
estableció en la metodología, este estudio se centrará en la resiliencia 
del suministro de GLP a nivel de almacenamiento primario en el nodo 
Coyhaique. Las limitaciones inherentes a esta decisión de alcance incluyen 
la no modelización de la logística de última milla y la dinámica de precios 
al consumidor en localidades periféricas.

No obstante, el prototipo validado sentará las bases para una robusta agenda 
de investigación futura. Las extensiones naturales del trabajo incluyen:
\begin{itemize}
    \item La evaluación de un portafolio más amplio de estrategias de 
    mitigación, como la ``Barcaza Energética'' o la mejora de puentes, ambas 
    propuestas en el informe de referencia.
    \item La incorporación de un modelo basado en agentes para analizar con 
    mayor profundidad el comportamiento competitivo y las posibles 
    estrategias de coordinación entre los distribuidores.
    \item La integración del modelo con un sistema de información en tiempo 
    real para evolucionar desde una herramienta de análisis estratégico a un 
    panel de control operativo para la gestión de emergencias.
\end{itemize}

En conclusión, este proyecto no solo busca responder una pregunta de 
investigación específica, sino también construir una plataforma metodológica 
y computacional escalable, con el potencial de convertirse en un activo 
estratégico para la seguridad energética de la Región de Aysén.