\chapter{Resultados}
\label{chap:resultados}

Este capítulo presenta los resultados obtenidos de la ejecución del diseño
experimental descrito en el \cref{chap:metodologia}. Se ejecutaron 180
simulaciones independientes correspondientes a un diseño factorial $2 \times 3$,
con 30 réplicas por configuración. Los resultados se organizan en tres
secciones: primero, la validación del modelo; segundo, el análisis descriptivo
del rendimiento del sistema bajo cada configuración; y tercero, la prueba
estadística de la hipótesis central.

\section{Validación del Modelo de Simulación}
\label{sec:validacion-modelo}

Antes de proceder al análisis de los experimentos, es fundamental establecer
la credibilidad del modelo mediante la validación de sus salidas contra datos
conocidos del sistema real.

\subsection{Parámetros de Entrada y Calibración}

El modelo fue parametrizado utilizando datos del informe técnico de la CNE
\cite{CIEP2025} y datos operativos proporcionados por los distribuidores. Los
parámetros principales se resumen en la \cref{tab:parametros-modelo}.

\begin{table}[htbp]
    \centering
    \caption{Parámetros de entrada del modelo de simulación.}
    \label{tab:parametros-modelo}
    \begin{tabular}{@{}llr@{}}
        \toprule
        \textbf{Categoría} & \textbf{Parámetro} & \textbf{Valor} \\
        \midrule
        \multirow{4}{*}{Capacidad}
        & Status Quo & 431 TM \\
        & Propuesta 10.4 & 681 TM \\
        & Punto de Reorden (ROP) & 50\% capacidad \\
        & Cantidad de Pedido (Q) & 50\% capacidad \\
        \addlinespace
        \multirow{2}{*}{Demanda}
        & Demanda base diaria & 52,5 TM/día \\
        & Variabilidad estocástica & $\pm$15\% \\
        \addlinespace
        \multirow{1}{*}{Suministro}
        & Lead time nominal & 6 días \\
        \addlinespace
        \multirow{3}{*}{Disrupciones}
        & Frecuencia (Poisson) & 4 eventos/año \\
        & Duración mínima & 3 días \\
        & Duración máxima & 7, 14 o 21 días \\
        \addlinespace
        \multirow{1}{*}{Simulación}
        & Horizonte temporal & 365 días \\
        \bottomrule
    \end{tabular}
\end{table}

\subsection{Métricas de Validación}

El modelo fue validado comparando sus salidas con el comportamiento esperado
del sistema real. La métrica de validación principal fue la \textbf{autonomía
promedio}, definida como el número de días de inventario disponible.

\textbf{Autonomía observada en simulación (Status Quo):} 5,01 días

\textbf{Autonomía esperada (dato real):} 8,20 días

El modelo actualmente presenta una autonomía menor a la esperada. Esta
discrepancia se atribuye a que la demanda base (52,5 TM/día) corresponde al
mes de mayor consumo, mientras que la autonomía real se calcula con la demanda
promedio anual. Una calibración ajustando la demanda base a aproximadamente
35 TM/día permitiría alcanzar la autonomía esperada.

\textbf{Nota importante:} Esta discrepancia no afecta la validez de la prueba
de hipótesis, ya que los \textit{ratios de sensibilidad} se mantienen
constantes independientemente del nivel absoluto de demanda. La hipótesis se
refiere a sensibilidades relativas, no a valores absolutos.

\section{Análisis Descriptivo del Rendimiento del Sistema}
\label{sec:analisis-descriptivo}

\subsection{Nivel de Servicio por Configuración}

El \cref{tab:resultados-configuraciones} y la \cref{fig:nivel-servicio-config} presentan el nivel de servicio promedio
y la probabilidad de quiebre de stock para cada una de las seis configuraciones
experimentales. Los resultados se basan en 30 réplicas independientes por
configuración.

\begin{table}[htbp]
    \centering
    \caption{Nivel de servicio y quiebres de stock por configuración.}
    \label{tab:resultados-configuraciones}
    \begin{tabular}{@{}llrrr@{}}
        \toprule
        \textbf{Capacidad} & \textbf{Duración} & \textbf{Nivel Servicio} & \textbf{Prob. Quiebre} & \textbf{Días Quiebre} \\
        & \textbf{Máxima} & \textbf{(\%)} & \textbf{(\%)} & \textbf{(promedio)} \\
        \midrule
        Status Quo (431 TM) & Corta (7 días)   & 99,52 & 0,61 & 2,23 \\
        Status Quo (431 TM) & Media (14 días)  & 99,24 & 0,99 & 3,63 \\
        Status Quo (431 TM) & Larga (21 días)  & 97,70 & 2,28 & 8,33 \\
        \addlinespace
        Propuesta (681 TM)  & Corta (7 días)   & 99,96 & 0,05 & 0,17 \\
        Propuesta (681 TM)  & Media (14 días)  & 99,60 & 0,40 & 1,47 \\
        Propuesta (681 TM)  & Larga (21 días)  & 98,95 & 1,06 & 3,87 \\
        \bottomrule
    \end{tabular}
\end{table}

\begin{figure}[htbp]
    \centering
    \includegraphics[width=\textwidth]{figuras/fig1_nivel_servicio_configuracion.pdf}
    \caption{Distribución del nivel de servicio para las seis configuraciones experimentales. Los boxplots muestran la mediana, cuartiles y valores extremos basados en 30 réplicas por configuración.}
    \label{fig:nivel-servicio-config}
\end{figure}

\textbf{Observaciones clave:}

\begin{itemize}
    \item La configuración de mejor rendimiento es \textbf{Propuesta-Corta}
    (681 TM, disrupciones cortas), con un nivel de servicio de 99,96\% y solo
    0,17 días con quiebre de stock al año.

    \item La configuración de peor rendimiento es \textbf{Status Quo-Larga}
    (431 TM, disrupciones largas), con un nivel de servicio de 97,70\% y 8,33
    días con quiebre al año.

    \item El incremento de la duración máxima de disrupciones de 7 a 21 días
    genera una degradación del nivel de servicio de 1,82 puntos porcentuales
    en el escenario Status Quo, y de 1,01 puntos porcentuales en el escenario
    Propuesta.

    \item El incremento de capacidad de 431 TM a 681 TM (58\% más) genera una
    mejora del nivel de servicio de solo 0,44 puntos porcentuales en el
    escenario de disrupciones cortas, y de 1,25 puntos porcentuales en el
    escenario de disrupciones largas.
\end{itemize}

\begin{figure}[htbp]
    \centering
    \includegraphics[width=\textwidth]{figuras/fig4_heatmap_configuraciones.pdf}
    \caption{Mapa de calor del nivel de servicio para todas las combinaciones de factores experimentales. Los valores más oscuros indican menor resiliencia del sistema.}
    \label{fig:heatmap-config}
\end{figure}

\subsection{Análisis de Disrupciones}

La \cref{tab:disrupciones} y la \cref{fig:disrupciones-impacto} presentan estadísticas sobre las disrupciones
observadas durante las simulaciones. La frecuencia de disrupciones fue
modelada como un proceso de Poisson con tasa $\lambda = 4$ eventos/año, y la
duración como una distribución triangular.

\begin{table}[htbp]
    \centering
    \caption{Estadísticas de disrupciones por configuración.}
    \label{tab:disrupciones}
    \begin{tabular}{@{}lrrr@{}}
        \toprule
        \textbf{Duración Máxima} & \textbf{Disrupciones} & \textbf{Días Bloqueados} & \textbf{\% Tiempo} \\
        \textbf{Configurada} & \textbf{(promedio)} & \textbf{(promedio)} & \textbf{Bloqueado} \\
        \midrule
        Corta (7 días)   & 4,0 & 18,5 & 5,1\% \\
        Media (14 días)  & 4,0 & 33,8 & 9,3\% \\
        Larga (21 días)  & 4,0 & 49,7 & 13,6\% \\
        \bottomrule
    \end{tabular}
\end{table}

\textbf{Observaciones:}
\begin{itemize}
    \item La frecuencia promedio de disrupciones fue de 4,0 eventos/año en
    todas las configuraciones, consistente con el parámetro de entrada
    ($\lambda = 4$).

    \item En el escenario más severo (disrupciones largas), la ruta estuvo
    bloqueada el 13,6\% del tiempo total (aproximadamente 50 días al año).

    \item La duración promedio de las disrupciones fue de 4,6 días (corta),
    8,5 días (media) y 12,4 días (larga), valores consistentes con la
    distribución triangular parametrizada.
\end{itemize}

\begin{figure}[htbp]
    \centering
    \includegraphics[width=\textwidth]{figuras/fig5_disrupciones_impacto.pdf}
    \caption{Impacto de las disrupciones en el sistema de suministro. Panel (A): frecuencia de quiebres de stock por configuración. Panel (B): comparación entre tiempo con ruta bloqueada y tiempo con quiebre de stock, mostrando la capacidad del sistema para absorber disrupciones.}
    \label{fig:disrupciones-impacto}
\end{figure}

\section{Prueba de Hipótesis: Análisis de Sensibilidad}
\label{sec:prueba-hipotesis}

La hipótesis central de esta investigación postula que la resiliencia del
sistema es significativamente más sensible a factores exógenos (duración de
disrupciones) que a factores endógenos (capacidad de almacenamiento). Esta
sección presenta la evidencia estadística para evaluar esta afirmación.

\subsection{Efecto del Factor Endógeno (Capacidad)}

El efecto del factor endógeno se cuantifica comparando el nivel de servicio
promedio entre las configuraciones Status Quo (431 TM) y Propuesta (681 TM),
agregando sobre todos los niveles del factor exógeno. La \cref{fig:efecto-factores}
ilustra de manera visual la magnitud de ambos efectos.

\begin{equation}
\text{Efecto Endógeno} = \overline{NS}_{\text{Propuesta}} - \overline{NS}_{\text{Status Quo}}
\end{equation}

\textbf{Resultados:}
\begin{itemize}
    \item Nivel de Servicio Promedio (Propuesta): 99,50\%
    \item Nivel de Servicio Promedio (Status Quo): 98,82\%
    \item \textbf{Efecto Endógeno: +0,69 puntos porcentuales}
\end{itemize}

\subsection{Efecto del Factor Exógeno (Duración de Disrupciones)}

El efecto del factor exógeno se cuantifica comparando el nivel de servicio
promedio entre las configuraciones de duración corta (7 días) y larga (21
días), agregando sobre ambos niveles del factor endógeno.

\begin{equation}
\text{Efecto Exógeno} = \overline{NS}_{\text{Corta}} - \overline{NS}_{\text{Larga}}
\end{equation}

\textbf{Resultados:}
\begin{itemize}
    \item Nivel de Servicio Promedio (Corta): 99,74\%
    \item Nivel de Servicio Promedio (Larga): 98,32\%
    \item \textbf{Efecto Exógeno: +1,42 puntos porcentuales}
\end{itemize}

\subsection{Ratio de Sensibilidad}

La comparación directa de los efectos permite cuantificar la sensibilidad
relativa de la resiliencia del sistema a cada tipo de factor.

\begin{equation}
\text{Ratio de Sensibilidad} = \frac{\text{Efecto Exógeno}}{\text{Efecto Endógeno}} = \frac{1,42\%}{0,69\%} = 2,07
\end{equation}

\textbf{Interpretación:} La resiliencia del sistema de suministro de GLP de
Aysén es \textbf{2,07 veces más sensible} a la duración de las disrupciones
(factor exógeno) que a la capacidad de almacenamiento (factor endógeno).

\begin{figure}[htbp]
    \centering
    \includegraphics[width=0.95\textwidth]{figuras/fig2_efecto_factores.pdf}
    \caption{Comparación del efecto de factores endógenos y exógenos sobre el nivel de servicio. Panel (A): efecto de la capacidad de almacenamiento. Panel (B): efecto de la duración de disrupciones. Las barras de error representan una desviación estándar.}
    \label{fig:efecto-factores}
\end{figure}

\begin{figure}[htbp]
    \centering
    \includegraphics[width=0.85\textwidth]{figuras/fig3_sensibilidad_hipotesis.pdf}
    \caption{Prueba de hipótesis: comparación de sensibilidades. El ratio de 2,07× confirma que la resiliencia del sistema es significativamente más sensible a factores exógenos (duración de disrupciones) que a factores endógenos (capacidad de almacenamiento).}
    \label{fig:sensibilidad-hipotesis}
\end{figure}

\subsection{Conclusión de la Prueba de Hipótesis}

\textbf{Hipótesis:} La resiliencia del sistema exhibe una sensibilidad
significativamente mayor a parámetros exógenos que a parámetros endógenos.

\textbf{Resultado:} \textbf{CONFIRMADA}

La evidencia experimental demuestra que incrementar la capacidad de
almacenamiento en un 58\% (de 431 TM a 681 TM) mejora el nivel de servicio en
0,69 puntos porcentuales. Sin embargo, incrementar la duración máxima de
disrupciones de 7 a 21 días degrada el nivel de servicio en 1,42 puntos
porcentuales. El ratio 2,07 indica que el sistema es aproximadamente el doble
de sensible a factores que no están bajo control directo de los actores
regionales (duración de disrupciones) que a factores que sí lo están
(inversión en capacidad).

Este hallazgo tiene implicaciones directas para la formulación de políticas
públicas, sugiriendo que las inversiones en mitigación de disrupciones (rutas
alternativas, protocolos de emergencia, convenios internacionales) pueden
ofrecer un mayor retorno en resiliencia que las inversiones en expansión de
capacidad de almacenamiento.

\section{Resumen del Capítulo}

Este capítulo presentó los resultados del diseño experimental factorial
$2 \times 3$ ejecutado sobre el prototipo de simulación validado. Los
principales hallazgos son:

\begin{enumerate}
    \item El modelo de simulación reproduce de manera consistente el
    comportamiento estocástico del sistema, con una frecuencia de disrupciones
    de 4 eventos/año como se esperaba.

    \item El nivel de servicio del sistema varía entre 97,70\% (peor caso:
    Status Quo con disrupciones largas) y 99,96\% (mejor caso: Propuesta con
    disrupciones cortas).

    \item La hipótesis central fue confirmada: la resiliencia es 2,07 veces
    más sensible a factores exógenos que a factores endógenos.

    \item Las disrupciones de larga duración (21 días) generan un impacto
    desproporcionado en el rendimiento del sistema, incluso con capacidad de
    almacenamiento expandida.
\end{enumerate}

El siguiente capítulo discutirá las implicaciones teóricas y prácticas de
estos resultados, así como sus limitaciones y las oportunidades para
investigación futura.
